\documentclass[czech,11pt,a4paper]{article}
\usepackage[utf8]{inputenc}
\usepackage{a4wide}
\usepackage[pdftex,breaklinks=true,colorlinks=true,urlcolor=blue,
  pagecolor=black,linkcolor=black]{hyperref}
\usepackage[czech]{babel}

\pagestyle{empty}

\renewcommand{\arraystretch}{1.3}

\begin{document}

\begin{center}
  {\Large --- Posudek vedoucího diplomové práce ---}
\end{center}

\vspace{.5cm}

\noindent \begin{tabular}{rp{.75\textwidth}}
  {\bf Diplomová práce:} & Implementace metody svazkového vyrovnání bloku pro určení prvků vnější orientace do programu GRASS GIS \\
  {\bf Student:} & Bc. Štěpán Turek \\
  {\bf Vedoucí:} & Ing. Martin Landa, Ph.D. \\
  {\bf Oponent:} & Dr. Yann Chemin \\
\end{tabular}

\vspace{1cm}

Cílem diplomové práce Štěpána Turka bylo navrhnout a implementovat
metodu svazkového vyrovnání bloku (BBA) do programu GRASS GIS se
zaměřením na zpracování dat pořízených z létajících bezpilotních
prostředků.
\\

Volba tématu byla do značné míry ovlivněna zkušeností studenta s
vývojem open source GIS aplikací, které získal během svého
bakalářského studia. Poslední semestr magisterského studia strávil v
rámci programu Erasmus Practical Placement v pracovní skupině \uv{GIS
  and Remote Sensing unit} ve Fondazione Edmund Mach (San Michele,
Trentino, Itálie). Téma práce bylo v předstihu konzultováno s vedoucím
pracovní skupiny Markusem Netelerem a její jádro vznikalo v rámci
této stáže. Student strávil tři měsíce v prostředí mezinárodního týmu,
který tvoří vedle zkušených pracovníků i řada studentů doktorského
studia. Prostředí, ve kterém student svoji práci zpracovával, jej
podle mého názoru pozitivně motivovalo a~výrazně se podepsalo na jejím
výsledku. V tomto ohledu bych rád poděkoval Markusovi Netelerovi a
celé jeho skupině za bezmála rodinný přístup, se kterým Štěpána Turka
přijali a po celou dobu jeho pobytu mu byli oporou.  \\

Text diplomové práce je napsán v anglickém jazyce a je členěn do tří
částí. V první části se autor věnuje teoretickým základům, ze kterých
čerpá v druhé analyticky orientované části práce. V třetí části se
věnuje samotné implementaci navrženého řešení. Následuje zhodnocení
výsledků na testovacích datech a poznámky k možnému budoucímu vývoji.
\\

Se Štěpánem Turkem jsem se blíže seznámil v rámci bakalářské práce,
kterou pod mým vedením předložil a úspěšně obhájil v červnu 2012. Od
té doby jsem s ním úzce spolupracoval na několika projektech. V letech
2012 a 2013 se zúčastnil prestižního mezinárodního projektu Google
Summer of Code zaměřeného na podporu mladých studentů z oblasti
informatiky a vývoje open source projektů. V rámci tohoto programu se
věnoval vývoji interaktivního nástroje pro síťové analýzy (2012) a
vývoji aplikace interaktivního rozptylogramu (2013) pro open source
GIS GRASS.
\\

Na závěr si dovolím konstatovat, že diplomová práce zcela naplňuje
očekávání, která jsem jako školitel do této práce vkládal. Diplomant
pracoval samostatně, v problematice se bez dřívější hlubší znalosti
zorientoval velmi rychle a i přes několik málo zaváhání a menších
tvůrčích krizí dokázal práci dotáhnout do úspěšného konce. V diplomové
práci se odráží dlouhodobý zájem a zkušenosti studenta s vývojem
nástrojů pro GIS.
\\

Cíl diplomové práce byl podle mého názoru splněn. Na základě
teoretického základu student navrhl prototyp ortorektifikace pro
systém GRASS, který by umožňoval zpracovat data pořízená z létajících
bezpilotních prostředků. Položil tak nutné základy k tomu, aby mohl
být celý projekt v budoucnu dotažen do funkčního celku. Doufám, že
diplomant bude mít možnost se věnovat tomuto projektu i nadále.
\\

Práce v tomto ohledu plně splňuje požadavky kladené na diplomovou
práci na studijním programu Geodézie a kartografie včetně všech
formálních náležitostí a podle mého názoru je i výrazně překračuje.
\\

Diplomovou práci Štěpána Turka doporučuji k obhajobě a hodnotím ji
stupněm

\begin{center}
{\bf --- A (výborně)  ---}
\end{center}

a doporučuji státnicové komisi, aby zvážila možnost podání návrhu
děkance fakulty na udělení pochvaly za vynikající zpracování, obhajobu
a přínos diplomové práce.

\vspace{2cm}

\noindent \begin{tabular}{lp{.28\textwidth}r}
V~Solanech dne 11.1.2014 & & \ldots\ldots\ldots\ldots\ldots\ldots\ldots \\
& & Ing. Martin Landa, Ph.D. \\
& & Fakulta stavební, ČVUT v Praze \\
\end{tabular}

\end{document}
