\documentclass[a4paper,12pt]{article}
\usepackage{a4wide}
\usepackage[T1]{fontenc}
\usepackage{lmodern}
\usepackage{textcomp} 
\usepackage[utf8]{inputenc}
\usepackage{xcolor}
\usepackage[czech,english]{babel}
\usepackage[pdftex, final]{graphicx}
% \usepackage[pdftex, final, colorlinks=true]{hyperref}
\usepackage{verbatim}
\usepackage{alltt}
\usepackage{paralist}
\usepackage{mdwlist}
\usepackage{subfig}
\usepackage[final]{pdfpages}
\usepackage{amsmath}
%\usepackage[hyphens]{url}
%\PassOptionsToPackage{hyphens}{url}

\usepackage{bibentry}
\makeatletter\let\saved@bibitem\@bibitem\makeatother
\usepackage[final,pdftex,colorlinks=false,breaklinks=true]{hyperref}
\makeatletter\let\@bibitem\saved@bibitem\makeatother
\usepackage[hyphenbreaks]{breakurl}

%%%%%%%%%%%%%%%%%%%%%%%%%
% pro podmineny preklad
% false je defaultně


% \newif\ifbc % Pouze do bakalářské práce
%  \bctrue

%%%%%%%%%% fancy %%%%%%%%%%%
\usepackage{fancyhdr}

\fancyhead[L]{CTU in Prague}

\setlength{\headheight}{16pt}

% \usepackage{stdpage}


%%%%%%%%%%%% rozmery %%%%%%%%%%%%%%%%%%
\usepackage[%
%top=40mm,
%bottom=35mm,
%left=40mm,
%right=30mm
top=40mm,
bottom=35mm,
left=35mm,
right=25mm
]{geometry}


\renewcommand\baselinestretch{1.3}
\parskip=0.8ex plus 0.4ex minus 0.1 ex

\newcommand{\klicslova}[2]{\noindent\textbf{#1: }#2}
\newcommand{\modul}[1]{\emph{#1}}
\author{Štěpán Turek}
% \pagecolor{darkGrey}
\newcommand{\necislovana}[1]{%
\phantomsection
\addcontentsline{toc}{section}{#1}
\section*{#1}

\markboth{\uppercase{#1}}{}
}

\newcommand{\ematr}[1]{
{\bf #1}
}

\newcommand{\evect}[1]{
{\bf #1}
}

\newcommand{\ehvect}[1]{
{\bf \widetilde{#1}}
}

\newcommand{\escal}[1]{
{\it #1}
}

\newcommand{\eucl}[1]{
{\bf R\textsuperscript{#1}}
}
\newcommand{\proj}[1]{
{\bf P\textsuperscript{#1}}
}

\newcommand{\efunc}[1]{
{\it #1}
}


\newcommand{\src}[1]{
{\it #1}
}

\newcommand{\term}[1]{
{\it #1}
}

%%%%%%%%%%%%%%%%%%%%%%%%%%%%%%
\begin{document}
\pagestyle{empty}

\begin{center}
%napisy

%%% ML: diplomka je v anglictine, potom by i uvodni strana mela byt v
%%% ang, viz
%%% http://geo.fsv.cvut.cz/proj/dp/2013/anna-kratochvilova-dp-2013.pdf
%%% ST: opraveno

%%% ML: u jmena si dopln titul (Bc.)
%%% ST:doplnen

\newcommand{\napisCVUT}{Czech Technical University in Prague}
\newcommand{\napisFS}{Faculty of Civil Engineering}
\newcommand{\napisObor}{Branch Geoinformatics}
\newcommand{\napisKatedra}{Department of Geomatics}
\newcommand{\napisVedouci}{Ing. Martin Landa, Ph.D.}
\newcommand{\napisAutor}{Bc. Štěpán Turek}
\newcommand{\napisDatum}{Mosty u Jablunkova 2013}
\newcommand{\napisNazevI}{Implementation of bundle block}
\newcommand{\napisNazevII}{ adjustment method for determination of}
\newcommand{\napisNazevIII}{exterior orientation into GRASS GIS}
\newcommand{\napisNazevAjI}{Implementace metody svazkového vyrovnání bloku}
\newcommand{\napisNazevAjII}{pro určení prvků vnější orientace do programu}
\newcommand{\napisNazevAjIII}{GRASS GIS}
\newcommand{\napisBakalarka}{Master’s thesis}
\newcommand{\napisPraha}{Mosty u Jablunkova 2013}


%
% prikazy
%\newcommand{\velka}[1]{\uppercase{#1}}
\newcommand{\velka}[1]{\textsc{#1}}
%
% 
\newif\ifpatitul
\patitultrue

\ifpatitul
{\Large\velka{\napisCVUT}}\\
\velka{\Large\napisFS}\\
\vfill
{\LARGE\velka{\napisBakalarka}}
\vfill
{\large\napisPraha\hfill\napisAutor}
\newpage
\fi%patitul



{\Large\velka{\napisCVUT}}\\
{\Large\velka{\napisFS}}\\
{\Large\velka{\napisObor}}
\vfill
\includegraphics[width=3cm]{logo_cvut_cb} %~
\vfill
{\Large\velka{\napisBakalarka}}\\
{\Large\velka{\napisNazevI\\
\napisNazevII\\
\napisNazevIII}}\\
{\large\velka{\napisNazevAjI\\
\napisNazevAjII\\
\napisNazevAjIII}}
\vfill
{\large%
Supervisor: \napisVedouci\\
\napisKatedra\\
\bigskip
\napisDatum\hfill\napisAutor}
\end{center}

\newpage
\definecolor{navodotisk}{RGB}{10,10,10}
\newcommand{\vlozZadani}{%
\Huge\textcolor{navodotisk}{\textsf{\textbf{ZDE VLOŽIT ORIGINÁLNÍ ZADÁNÍ}}}%
}
%%%\includepdf[picturecommand={\put(100,200){\vlozZadani}}]{zadani}
 % resi si zalomeni sam

\begin{abstract}

\bigskip

\klicslova{Klíčová slova}{GIS, GRASS}

\end{abstract}

\selectlanguage{english}
\begin{abstract}

\bigskip

\klicslova{Keywords}{GIS, GRASS}

\end{abstract}
\selectlanguage{english}


\newpage
\newcommand{\odsaditodzhora}{\hskip1pt\vfill}

\odsaditodzhora
\noindent Declaration of authorship

I declare that the work presented here is, to the best of my knowledge and
belief, original and the result of my own investigations, except as acknowledged.
Formulations and ideas taken from other sources are cited as such.

December 21, 2013

\begin{flushleft}
\begin{tabular}{cp{0.3\textwidth}c}
In Mosty u Jablunkova, 
& 
&
..................................
\\
&&
(author’s signature)
\end{tabular}

\end{flushleft}
\newpage

%%% -> ML: chybi ;-)

\odsaditodzhora
\noindent Poděkování


\newpage

\newpage
\tableofcontents


\newpage
\pagestyle{fancy}

\necislovana{Introduction}

In recent decades tremendous development of information technology has fundamentally changed many disciplines.

Photogrammetry has successfully taken advantages of this development, and nowadays 
it is using methods, which allows to process data 
in huge scales achieving high accuracy. Some of the currently used 
methods has been known for several decades, however technology was not so advanced in order to be employed 
 in practice. 

Information technology has caused blurring of borders between disciplines. 
Photogrammetry has been affected by computer vision, which evolved independently until 2000's. After that it is possible 
to see growing trend of combining knowledge from both fields. 

%%% ML: seznam zkratek je lepsi generovat automaticky a stejne tak
%%% pouzivat v textu, viz
%%% ftp://ftp.tex.ac.uk/tex-archive/macros/latex/contrib/acronym/acronym.pdf,
%%% ted uz to ale nech tak jak to je, spis poznamka az budes psat
%%% nejaky podobny text priste ;-)

Both photogrammetry and computer vision are dealing with information recorded in photo. Computer vision 
methods are employed in robotics, dealing with determination of robot position and retrieving structure 
of environment which surrounds the robot.
Photogrammetry is used to produce digital terrain models (DTM), precise structure of objects or orthophotos. 

%%% ML: veta "Nowadays, thanks to current state of the art software,
%%% ..." zni divne...

The core of photogrammetry is quickly moving  from  hardware towards software. In past it was necessary to have special 
expensive equipment to perform photogrammetric measurement. Nowadays, thanks to current state of the art software, 
 it is possible to use higher level digital cameras to perform precise photogrammetric measurements. 
 Currently it is even possible to process data from optically inaccurate low level cameras.
 The inaccuracy is compensated by processing big number of photos bringing redundancy, 
 which positively affects precision of retrieved information.
 
As a consequence of these changes, photogrammetry has not longer been domain of few professionals but it has 
become used by much wider audience. 

%%% ML: Geographical -> Geographic (jiz opraveno)

Geographic information system (GIS) is software which is used for storing and processing information,
with spatial component. For instance  GIS is employed on backed of map portals (e.g. Google Maps) therefore nearly 
everybody uses its products nowadays. 
The GIS systems are also used for analyzing spatial phenomenons, which
are utilized by decision makers (politicians, managers), to get in-depth knowledge about the 
phenomenon they are dealing with.

%%% ML: slovu 'blackbox' by mozna slusely uvozovky

%%% ML: e. g. -> e.g. --- viz http://en.wiktionary.org/wiki/e.g. (jiz opraveno)

With growing importance of software there comes a question of licenses. There are two 
main branches: proprietary software and free software. Most of proprietary software is not available in 
form of source code. The license of proprietary software allows to use it under certain terms and usually user 
has to buy it. Proprietary software works like blackbox, because user has no chance to investigate the source 
code and realize what is going on inside it. Therefore user can only believe to software provider
that it is really doing what the provider advertises. 
For instance it can be problem when some data are analyzed
because the user loses control 
over the data in processing chains which uses proprietary software in some steps.
This also posses very serious security threads, because the software can do some hidden activity against user
e.g. obtaining personal data or tracking user activity.

%%% ML: "a active communities" - to by melo byt bez clenu (mluvis v
%%% mnoznem cisle) bych rekl (v kazdem pripade by tam melo byt 'an')

Despite the fact that current complexity of software is so huge,
that single human is not capable to really understand every piece of software, which uses,
free software does not posses disadvantages of proprietary softwares.
Transparency of the free software makes its source code subject to public control 
carried out by community of developers which are gathered around the project.
All popular free software projects have such a active communities, therefore it is
very unlikely that same malicious code would be included into the project.
Moreover if such a case would happen, it would not take a long time until such a change 
would be spotted by someone from the community.

GRASS GIS  (Geographic Resources Analysis Support System) is one of the free GIS software. 
Besides many other features, GRASS is capable to create orthophoto from aerial photos.
GRASS orthorectification processing chain was implemented in 1990s.
The processing chain was focused on  processing of photos taken by rigorous 
photogrammetric aerial missions performed by manned aircrafts, which was 
major method of obtaining orthophotos in those days. 

Nowadays there is a strong demand for  providing simple and general solution 
which is able to produce orthophoto, structure of objects or 
DTM. Current processing chain is not able to meet up such a demand.
 The thesis tries to analyze current state of orthorectificaion in GRASS,
explore and implement methods of photogrammetry and computer vision in order 
to create such a solution. 

\section{Theoretical part}

%%% ML: tohle vypada jako chyba, kapitola "Theoretical part" je
%%% prazdna, nemela byt "Essential principle" podkapitola (subsection)? 

\section{Essential principle}
\label{sec:ess_princip}

Both photogrammetry and computer vision are dealing with retrieving structure of scene from group of photos 
 recording it. Structure retrieval methods are based on pinhole camera model, which 
describes relation of \term{object point} in 
3D camera coordinate system and projected 2D \term{photo point} in photo coordinate system.
Parameters which characterizes pinhole camera model are called \term{interior orientation}.

\begin{figure}[h]
    \centering
    \includegraphics[scale=0.3]{figures/pinhole_camera.png}
    \caption{Pinhole camera model}
\end{figure}



\term{Image plane} is represented by digital camera image sensor, which  creates photo converting light into electronic signal.
Then the signal is processed inside the camera giving output in form of photograph, which is stored in a digital format. 
Image sensor is divided into cells, which measure intensity of fallen light on their surface. The pixel value in final 
photograph is based on the measured intensity of cell or group of cells. Photo created by image sensor is inverted. This kind 
of photo is called \term{negative} in analog photography.  Usually inversion is removed during processing inside 
camera. The resulting non inverted photo is called \term{positive} in analog photography. 

\begin{figure}[h]
    \centering
    \includegraphics[scale=0.4]{figures/pinhole_camera_inversion.png}
    \caption{Pinhole camera photo inversion \cite{mellish2005pinhole}}
\end{figure}

% ML: \term{} na konci vety vytvari NEZADOUCI MEZERU, to se da
% odstranit v definici znakem '%'

All light \term{rays} going from the object points and falling on image sensor are going through the same point in pinhole camera model.
The point is called \term{projective center}. The ray which is perpendicular to the image sensor plane is called principal ray and the point 
where the \term{principal ray} intersects the image sensor plane is called \term{principal point}.

Pinhole camera model defines two coordinate systems. 2D  \term{photo coordinate system} is coincident
with plane of positive.
Coordinates of photo point are expressed in this system. Another one is 3D \term{camera coordinate system}
where the object point is located. The system originates in projective center, Z coordinate 
 is coincident with projective ray and X, Y axes being in parallel to x, y axes of photo system.

 
As a consequence of dimensional reduction caused by 
3D scene projection onto 2D plane, some information is lost.
If 3D coordinates of point are known it is able to reproject it on the image sensor plane. However because 
of the dimensional reduction,  only the ray can be reconstructed from the 2D photo point,
3D coordinates of the point cannot be determined.
 
In order to solve the ambiguity of object point position on the ray another information is needed.  
%%% ML: "a additional" asi bez clenu, jinak je spravne 'an'
Such a additional information can provide another photo from different position, which contains same point. 
If positions of photos 
 are known (this is called \term{exterior orientation}), it is possible 
to reconstruct 3D coordinates of the point, because it lies on the intersection of the photo rays.

%%% ML: predpokladam, ze obrazky bez uvedene reference jsou puvodni

\begin{figure}[h]
    \centering
    \includegraphics[scale=0.2]{figures/object_point_intersection.png}
    \caption{Object point reconstruction from intersection of two photo rays.}
    \label{fig:obj_intersec}
\end{figure}

Real positions of points and exterior orientation of cameras are expressed in object coordinate system. 
Exterior orientation describes relation between object coordinate system and camera coordinate system. 
Relation between point in object 
space and point in photo space can be expressed by combining interior and exterior orientation together.

In this thesis, there are introduced and used such a methods leading to retrieval of the scene structure from the 
the set of photos, which are based on such a simple principle of ray intersection! Computer vision calls it structure from motion 
and photogrammetry uses it for creation of orthophoto, DTM or structure of object.

%%% ML: spravnejsi by bylo napsat DMT v kombinaci s ortofotem (drap
%%% over)

\begin{figure}[h]
    \centering
    \includegraphics[scale=0.3]{figures/dtm.jpg}
    \caption{Digital terrain model \cite{kbosak2010bezmiechowa}.}
\end{figure}

%%% ML: misto Mapy.cz bych asi uvedl mezinarodni projekt anebo v
%%% poznamce pod carou vysvetlit co je Mapy.cz

Orthophoto is created in 
orthorectification process, which is transformation from perspective projection into orthogonal projection.
Unlike non-orthorectified pohotography the orthophotos can be merged together into seamless map. Recently 
orthophoto has been widely used by map services like Google Maps or Mapy.cz. Ortho\-rectified photo 
has same scale in all parts, therefore it is possible to measure true distances as in the map. During orthorectification process distortion 
caused by tilt of camera (if photo is not vertical) together with elevation differences are removed.

%%% ML: pokud je popisek obrazku moc dlouhy, tak by mel byt pro sekci
%%% 'seznam obrazku' zkracen \caption[kratky]{dlouhy}

\begin{figure}[h]
    \centering
    \includegraphics[scale=0.2]{figures/orthophoto.png}
    \caption{On the left picture there is showed effect of elevation differences 
    which arises from perspective projection of pinhole camera model. 
    On the right picture there is depicted orthogonal projection of orthophoto.}
    \label{fig:ortho}
\end{figure}


\subsection{GRASS GIS}

%%% ML: hodila by se reference na GNU GPL a mozna i na webove stranky
%%% projektu GRASS

GRASS (Geographic Resources Analysis Support System) GIS is open source GIS software cabable of processing raster, vector and imagery data. 
GRASS belongs among one of 
the longest active software project. The development of GRASS began by U.S. Army CERL  laboratory
in first half of 1980s. In 1990s U.S. Army gradually ceased developing GRASS GIS. Luckily it was decided to 
release source code as public domain, therefore the development could be taken over by volunteers. In the 1999 license 
of GRASS was changed into GNU GPL v2. Currently GRASS GIS is developed by the GRASS Development 
Team, which is comprised of developers spread world wide. 

%%% ML: reference na QGIS a ArcGIS nebo jako poznamka pod carou (?)

%%% ML: "a new users", bez clenu nebo "the new users" (?)

Individual functionalities of GRASS are implemented in form of small software packages called "modules", which 
can be combined to solve complex tasks.
Unlike other GIS systems as QGIS or ArcGIS, it takes 
more time for a new users to start understand the GRASS philosophy. GRASS developers 
have been aware of this problem, therefore they decided to develop graphical user interface, which 
has evolved dynamically.

%%% ML: "so called" GIS database (?)

Data in GRASS are stored in GIS Database. GIS Database is composed of locations. Every location 
defines its projection and all data, which belongs to the location, must be reprojected to 
the projection of location. The location comprises from mapsets, which are used for organization 
of data into consistent units. 

%%% ML: pro citelnost textu by mozna bylo dobre slova location, mapset
%%% ci comp. region pri prvnim pouziti zvyraznit ({\em }) - osobni
%%% nazor (low priority)


%%% ML: nektere moduly pro praci s vektorovymi daty maji prepinar pro
%%% akcepovani regionu, ale zde bych to neuvadel, neni to pro praci
%%% dulezite

%%% ML: extend -> extent (jiz opraveno, snad jsem nic neprehlednul)

Another important element of GRASS is computational region. Region stores information about geographical
extent, which is defined by four values (north, south, east, west) of bounding box and 
two values, which defines south-north and west-east resolution. Properly set region is 
very important for raster data processing, because nearly all except import modules provides computation 
on currently set region. Vector data are processed in whole extent regardless region.

\subsection{UAV}

%%% ML: myslim, ze se nejvic UAV vyrabi v Izraely, pouziti pro
%%% vojenske ucely je nejvice viditelne u U.S. Army, ta Izraelska je
%%% snad pouziva jeste vic, to jenom na okraj

Unmanned aerial vehicle (UAV also known as drone) is aircraft without human pilot. Recently UAVs are ingloriously known for
their heavy military employment especially by USA in Afghanistan and Pakistan. These military drones are equipped with state of the art 
technology and their construction is similarly complex as small aircraft.  
The number of military UAVs is quickly raising and nowadays USA trains more UAV operators than fighter plane pilots.

%%% ML: prize -> price (jiz opraveno)

There is also exponentially growing sector of civil drones, which is very broad category ranging
from big military like size drones to small ones in size of few decimeters. There are several types of drones e.g. airplanes,
helicopters, multicopters, paraglides, airships, balloons and others. 
Very promising category of drones are multirotor drones (e.g. hexacopters or octocopters), because they have simple construction and 
even ordinary users are able to learn control them very quickly. Together with steeply decreasing prises of octocopters it can be expected 
that this kind of drone will become commonly used for many civil purposes.

In photogrammetry cheap UAVs can be successfully employed in mapping missions for creation of DTM/structure retrieval and orthophoto.
Main advantage of the UAV mission are cost savings compared to mission which is performed by manned aircraft.


\begin{figure}[h]
    \centering
    \includegraphics[scale=0.8]{figures/octocopter.jpg}
    \caption{Octocopter \cite{boe2013octocopter}}
    \label{fig:sample_figure}
\end{figure}

%%% ML: pred finalnim tiskem zkontrolovat rozumne pozicovani obrazku a
%%% hlavne zacatek kapitoli na konci stranky, da se vyresit zvetseni
%%% mezery (\vskip) anebo vynucenim konce stranky (\newline)

\subsection{Least squares}
\label{sec:least}

%%% ML: v kapitole nejsou uvedeny ZADNE reference, nepredpokladam, ze
%%% jsi ji psal z patra bez pouziiti zadne literatury...

In linear algebra, commonly solved equation is:

\begin{equation}
\label{eq:linear_eq}
\ematr{A}\evect{X} = \evect{L} 
\end{equation} 


%%% ML: \evect{} podobny problem jako u \term{} - VYNUCENA MEZERA, coz
%%% dela neplechu na konci radky (mezera pred teckou), podobnym
%%% problem budou trpet i dalsi prikazy, chtelo by je opravit
%%% (%) a zkontrolovat, zda nekde mezera naopak nechybi

which can be interpreted as:
\begin{itemize}
\item \evect{L} - vector of measurements
\item \evect{X} - vector of parameters, which values are searched to solve the system
\item \ematr{A} - it is matrix of the linear cost function  coefficients,  where row represents one measurement 
		  and column represents one parameter of cost function. Linear cost function $\efunc{f_{i}(\evect{X})}$
		  represents relationship between measurements \evect{L} and searched parameters \evect{X}. 
		  Least square method calls it design matrix. 
\end{itemize}

%%% ML: prvni veta zni divne...

If matrix \ematr{A} is 
non singular matrix  and vector \evect{b} is non zero vector, then there exists exactly one solution 
for vector of parameters  \evect{X} to satisfy the equation. Non singular matrix is always square matrix, 
which has all column vectors and row vectors linearly independent. 

%%% ML: zvyraznit jednotlive polozky (?), napr. {\em gross error}

Basically every measurement is burden with some error. The errors can be devided into these main categories:  
\begin{itemize}
\item gross error - error caused by human factor or fault of instrument. If many measurements are repeated
usually it is possible to identify measurements with this errors because they are big outliers from the other measurements, which are not affected with this 
kind of error.
\item systematic error - this error always has same effect (value) on measurement.  Usually effect of this error can be 
eliminated by choosing right measurement method, calibration of instrument or taking it into account in  mathematic model.
\item random error - this error is unavoidable because it is impossible to know perfectly all factors which has impact 
on a measurement. The random errors follows normal distribution which is represented with probability density function of this shape:

%%% ML: TODO

TODO picture


The normal distribution is characterized by two parameters, which are standard deviation 
$\escal{\sigma}$ and mean $\escal{\mu}$. 
If measurements are burden only with random error the mean represents accurate value and standard deviation represents 
dispersion level of measurements. 
Theoretically  if measurements of some parameter would be repeated infinite times and the measurements would be influenced only by random errors,
accurate values could be computed as mean.
It follows that it is possible to get closer to the accurate value by repeating of measuremnts, which 
leads into solving overdetermined systems. Another important implication is that if same number of measurement is perfromed by more accurate 
instrument, the mean of measurements is statistically closer to the accurate value than the mean of measurements perform with less precise instrument.
Measurements of different accuracies, can be combined together by weighted mean
with weight of measurement $\escal{p}_{i}$ defined as:
\begin{equation}
\label{eq:wmean}
\escal{p}_{i} = \frac{\escal{const}}{\escal{\sigma}_{i}^{2}}
\end{equation} 
where \escal{const} is arbitrary chosen number which is used for calculation of all weights 
and $\escal{\sigma_{i}}$ is standard deviation of the instrument used for the measurement.
\end{itemize}

The overdetermined systems are solved in many fields which deals with measurements, where the precision matters (e.g. 
surveying, photogrammetry etc.).
As consequence of overdetermination, the design matrix \ematr{A} becomes rectangular with more rows than columns 
and therefore the solution does not exist for this linear equations system \eqref{eq:linear_eq}.
Only if the measurements would be perfect it would be possible to find such a parameters vector \evect{X}, to 
solve the system of linear equations \eqref{eq:linear_eq}. 

Due to measurement imperfections it is not possible to find such a parameters of vector \evect{X} to make difference \evect{v} between measured values and 
result of cost function using the parameters equal to zero.

\begin{equation}
\label{eq:least_v}
\evect{v} = \evect{L} - \ematr{A}\evect{X}
\end{equation} 



Hence it is needed to define another condition, which allows to find optimal solution of the 
overdetermined system.

\begin{figure}[h]
    \centering
    \includegraphics[scale=0.2]{figures/squares.png}
    \caption{In this case least squares method finds such a line,
    which minimizes sum of squares areas. }
    \label{fig:squares}
\end{figure}

The least square method finds such a values of parameters vector \evect{X}, which 
follows condition of minimum sum of square differences \evect{v}: 

\begin{equation}
\evect{v}^{T} \evect{v} = min
\end{equation} 

The minimized equation can be rewritten as:

\begin{equation}
\begin{split}
\evect{v}^{T} \evect{v} &= (\evect{L} - \ematr{A}\ematr{X})^{T} (\evect{L} - \ematr{A}\ematr{X}) \\
&= \evect{L}^{T} \evect{L} - 2 \evect{L}^{T} \ematr{A} \evect{X} + \evect{X}^{T} \ematr{A}^{T} \ematr{A} \evect{X}
\label{eq:least_be_part}
\end{split}
\end{equation} 

According to calculus notation least squares method searches for global minimum of the square difference function. 
The global minimum of the function can be obtained using partial derivative of parameters in the cost function.
In this case derivatives are simple because it is supposed that cost functions are linear, therefore 
its square forms  second degree polynomial e.g. parabola in \eucl{2}, paraboloid in \eucl{3} etc. All these functions has
only one point, where all partial derivatives are equal to zero, regardless dimension of the function. It is 
exactly the global minimum, which the least squares method searches for!

The partial derivative of minimized function can be written as:
 
\begin{equation}
\frac{\partial \efunc{f} (\evect{v}^{T} \evect{v}) }{\partial \evect{X}} = -2\ematr{A}^{T} \evect{L} + -2\ematr{A}^{T}\ematr{A} \evect{X} 
\label{eq:least_part}
\end{equation} 

Giving this equation equal to zero, the global minimum can be find:
\begin{equation}
-2\ematr{A}^{T} \evect{L} + -2\ematr{A}^{T}\ematr{A} \evect{X} = 0 
\end{equation} 

The least square determines values of parameters vector \evect{X}, which can be expressed from previous equation:
\begin{equation}
\evect{X} = (\ematr{A}^{T} \ematr{A})^{-1} \ematr{A}^{T} \ematr{L}
\end{equation}

which is essential equation of the least squares method. The equation can be solved simply by means of linear algebra.

The equation supposes that all measurements have same weight (accuracy). The least squares method can be extended to support weights.
Weight matrix \ematr{P} contains measurements weights on the diagonal with off-diagonal being zero elements if
the measurements are not correlated.
Weight matrix row of individual weight determines a measurement in design matrix \ematr{A} where the weight belongs.

The equation, which is minimized by least squares is extended by weight matrix:
\begin{equation}
\evect{v}^{T}  \ematr{P} \evect{v} = min
\end{equation}

therefore the least square equation is altered into this form:
\begin{equation}
\label{eq:least_sq}
\evect{X} = (\ematr{A}^{T} \ematr{P} \ematr{A})^{-1} \ematr{A}^{T} \ematr{P} \ematr{L}
\end{equation}

\subsubsection{Non-linear least squares}
\label{sec:non_least}
So far it was supposed that the cost function is linear thus global minimum can be found simply by means of linear algebra  \eqref{eq:least_sq}.
Unfortunately it is very common that the relationship between parameters and 
measured values is not linear, hence it is needed to adapt linear least squares equations for non-linear
cost functions.

Unlike linear least squares, which square of difference \label{eq:least_be_part}
is represented by function  (parabola, paraboloid...) with just one point where all
partial derivatives are being zero, the non-linear cost function can have many such a points e.g. other local minimas. 

This problem of non-linearity can be solved by linearization of non-linear equations.
The main idea of linearization process is to approximate the cost function with first degree Taylor polynomial,
which is linear and therefore it is possible to use linear least square equation \eqref{eq:least_sq} 
to solve non-linear problem. 

\begin{figure}[h]
    \centering
    \includegraphics[scale=0.2]{figures/taylor.png}
    \caption{Function $\efunc{f}_{i}$ is approximated by first degree Taylor polynom $\efunc{T} ^{\efunc{f}_{i}}$,
    which forms line. Symbol $\Delta$ represents an aproximation error.}
    \label{fig:taylor}
\end{figure}


Approximation of the cost function with a first degree Taylor polynomial can be written as:
\begin{equation}
\efunc{T} ^{\efunc{f}_{i}, \evect{X}_{0}}_{1} (\evect{X}) = 
	      \efunc{f}_{i}(\evect{X}_{0}) + 
               \frac{\partial}{\partial {\escal{X}_{0_{1}}}} \efunc{f}_{i} (\evect{X}_{0}) (\evect{X} -  \evect{X}_{0}) 
\ ... \ + \frac{\partial}{\partial {\escal{X}_{0_{n}}}} \efunc{f_{i}}(\evect{X}_{n}) (\evect{X} - \evect{X}_{0}) 
\end{equation}

%%% ML: pozor na konec stranky (projit az na konci pred finalnim
%%% tiskem)

%%% ML: nemel by to byt clen? "is a point"

where:
\begin{itemize}
\item $\evect{X}_{0}$ is point where the Taylor polynomial touches function, which it approximates. E.g. first degree Taylor polynomial 
      is line, which is tangent of function at point $X_{0}$ in \eucl{2}, tangent plane in \eucl{3} and  
      tangent hyperplane in  \eucl{n}.
\item \evect{X} is point where a Taylor polynomial returns approximate value of function $\efunc{f}_{i}$.
\end{itemize}

Least squares Taylor polynomial can be written as:

\begin{equation}
\efunc{T} ^{\efunc{f}_{i}, \evect{X}_{0}}_{1} (\evect{X}) =  \efunc{f}_{i}(\evect{X}_{0}) + \escal{a}_{0} \escal{dx}_{0} \ ... \ + \escal{a}_{n} \escal{dx}_{n} 
\end{equation} 

in the matrix form with formal substitution $\evect{L_{X_{0}}} = \efunc{f}_{i}(\evect{X}_{0})$:
\begin{equation}
\label{eq:expanded_taylor}
\efunc{T} ^{\efunc{f}_{i}, \evect{X}_{0}}_{1} (\evect{X}) = \evect{L_{X_{0}}} + \ematr{A}\evect{dx}
\end{equation} 

And analogous equation to \eqref{eq:least_v} can be derived:
\begin{equation}
\evect{v} = \evect{L} - \efunc{T} ^{\efunc{f}_{i}, \evect{X}_{0}}_{1} (\evect{X})
\end{equation} 
 
Taylor polynomial member can be expanded by \eqref{eq:expanded_taylor}:
\begin{equation}
\begin{split}
\evect{v} &=  \evect{L_{X_{0}}} - \evect{L} - \ematr{A}\evect{dx} \\
          &= \evect{l} - \ematr{A}\evect{dx}
\end{split}
\end{equation}

If the differences vector is substituted into least squares minimization criterion:
\begin{equation}
\label{eq:min_cond}
\evect{v}^{T}  \ematr{P} \evect{v} = min
\end{equation}

the essential equation for non linear least squares can be derived 
similarly to linear least squares:

\begin{equation}
\begin{split}
\label{eq:non_least_sq}
\evect{dx} &= (\ematr{A}^{T} \ematr{P} \ematr{A})^{-1} \ematr{A}^{T} \ematr{P} \ematr{l} \\
\evect{X} &=  {\evect{dx} -  \evect{X_{0}}} \\
\end{split}
\end{equation}

%%% ML: prize -> price (jiz opraveno)

Comparing the equation to the linear least squares equation \eqref{eq:least_sq},
 it is slightly different. Instead of parameters vector \evect{X} there is used 
parameters vector of differences \evect{dx} and measurement vector \evect{L} is analogous to vector \evect{l}. In this two minor 
substitutions there are hidden major limitations of the non-linear least squares method, 
where the price for approximation is paid. 

As it was already mentioned that the Taylor polynomial requires some point $\evect{X_{0}}$, where it is tangent to 
approximated function.
This approximation is good in close surroundings of point $\evect{X_{0}}$, however as the distance 
grows the approximation error $\Delta$ may increase significantly, which can be clearly seen in figure \ref{fig:taylor}.

Therefore, unlike linear least square method, non-linear least squares requires initial 
values of parameters \evect{X}. If the initial values are not accurate 
enough, it is possible to iteratively run linear least squares subsequently using adjusted parameters vector \evect{X}
from previous iteration to eventually approach global minimum of the cost function. 
Therefore the closeness of the initial point $\evect{X_{0}}$ to global minimum affects how many iterations (speed of convergence)
are needed to achieve optimal solution (global minimum).

Unfortunately the non-linear least square method has another one, even more serious pitfall, which 
does not guarantee that it will iterate towards global minimum at all!
In this case, the found parameters \evect{X} are completely wrong because 
it does not satisfies minimum squares difference condition \eqref{eq:min_cond}. If the initial values are 
inaccurate, the least squares can iterate toward other points with zero derivation (e.g. local minimas).  
This is caused by run-off of non-linear functions, which has more points with zero derivative
and therefore result depends on the initial point given by $\evect{X_{0}}$  of non-linear least squares iteration.


\subsubsection{Least squares properties}

It was proved that above derived least square method is solution of Gauss-Markoff model, which 
says if measurements \evect{L} are random with known covariance matrix $\ematr{\Sigma}_{l}$  and vector of parameters \evect{X} is non-random, then
vector \evect{dx}  has the lowest variance and it is unbiased. If density functions of measurements are normally 
distributed then values obtained from cost functions using  computed parameters lies in maximum of the density
function. In other words they posses maximum likehood.

The weight matrix is calculated form covariance matrix in similar way as weights in weighted mean \eqref{eq:wmean}:
\begin{equation}
\ematr{P} = \frac{\escal{const}}{\escal{\Sigma}_{i}^{2}}
\end{equation} 

\escal{const} is chosen arbitrary. It can be selected as unit a priori weight standard deviation $\sigma_{0}$. 


Unbiased unit weight standard deviation $\hat{\sigma}_{0}$ can be also estimated a posteriori as:
\begin{equation}
\escal{\hat{\sigma}}_{0}^{2} = \frac{\evect{v}^{T} \ematr{P}  \evect{v}}{\escal{r}}
\end{equation} 

where \escal{r} is redundancy, which is difference of columns number from rows number of design matrix \ematr{A}.

Covariance of calculated parameters $\ematr{\Sigma}_{x}$ can be get from:
\begin{equation}
\ematr{\Sigma}_{\hat{x}} = \escal{\hat{\sigma}}_{0}^{2} (\ematr{A}^{T} \ematr{P} \ematr{A})^{-1}
\end{equation} 

and a posteriori covariance matrix $\ematr{\Sigma}_{\hat{l}}$ of measurements can be obtained from:
\begin{equation}
\ematr{\Sigma}_{\hat{l}} =  \ematr{A} \ematr{\Sigma}_{\hat{x}} \ematr{A}^{T}
\end{equation} 

\subsubsection{Free network least squares}
\label{sec:free_net_least}

%%% ML: "datum deficiency" by chtelo zvyraznit (\em) - pouze osobni nazor

In geodesy, it can happen that columns of design matrix \ematr{A}
are not linearly independent. This can be caused by missing information
about coordinate system in the design matrix. This issue is called datum deficiency.
Datum deficiency makes normal matrix $ \ematr{A}^{T} \ematr{P} \ematr{A}$ non invertible and therefore 
least squares equation \eqref{eq:non_least_sq} cannot be solved.
In order to avoid the datum deficiency issue it is needed to keep some parameters fixed, 
which means that columns of the fixed parameters are excluded from design matrix\ematr{A}.

Photogrammetry deals with three dimensional spaces, which requires fixing of at least  
two points and another point with one coordinate to avoid datum deficiency problem, because
three dimensional coordinate system is defined by three coordinates of shift vector, three rotation angles and scale factor. 

If there are not enough fixed points, the network has to be adjusted as a free network.
One of possible methods how to adjust free network is calculation of pseudo inverse matrix of the normal matrix instead of
inversion. Another option is adding inner constraints to the design matrix.

\subsection{Short introduction to photogrammetry}

Essential information, which must be known to obtain any photogrammetric
 product (e. g. orthophoto or DTM) from group of photos is their interior orientation and exterior orientation.
Exterior and interior orientation together describes relation between object point and
it's corresponding photo point.

\subsubsection{Interior orientation}

Interior orientation describes relation between measured 2D photo coordinates 
and corresponding 3D in camera coordinate system which is based on pinhole camera model described in \ref{sec:ess_princip}.

Photogrammetric camera coordinate system has origin in projective point,
z axis heading out of the scene  and x, y axes being parallel to photo plane. Directions of 
x, y axes are selected in way to be the camera system right handed.

\begin{figure}[h]
    \centering
    \includegraphics[scale=0.3]{figures/photogrammetric_model.png}
    \caption{Photogrammetric camera coordinate system in arbitrary object coordinate system.}
    \label{fig:ph_model}
\end{figure}

Interior orientations parameters describing pinhole camera model are:
\begin{itemize}
  \item Focal length \escal{f} - it is distance of projective center from principal point.
  \item Principal point coordinates $(\escal{x}^{ph}_{pp}, \escal{y}^{ph}_{pp})$ - In ideal case photo system position of principal point would be exactly in the middle 
	of the photo.  However in most cases there is some shift of principal point from the photo center due to
	camera construction inaccuracies.
\end{itemize}

Camera optical system is imperfect, therefore it deviate from the ideal pinhole model which 
 causes distortion of photo.
The effect of distortion can be reduced by applying distortion corrections. Commonly 
used model for distortion correction is Browns model \cite{brown1966distortion}.
The model defines these two types of distortion:
\begin{itemize}
  \item Radial distortion - it depends on distance from principal point. Cause of the distortion is spherical shape of the 
  lens in camera optical system.
  \item Tangential distortion - it is perpendicular to direction of radial distortion. It is caused by 
       imperfect alignment of lenses in camera optical system.
\end{itemize}

Mathematically brown model is expressed by these equations:

\begin{equation}
\label{eq:undistort}
\begin{split}
\escal{x}^{u} =& (\escal{x}^{ph} - \escal{x}^{ph}_{pp})(1 + \escal{K}_{1}r^2 + \escal{K}_2r^4 + ...) + \\
&(\escal{P}_1(\escal{r}^2 + 2(\escal{x}^{ph} - \escal{x}^{ph}_{pp})^2) + 2\escal{P}_2(\escal{x}^{ph} - \escal{x}^{ph}_{pp})(\escal{y}^{ph} - \escal{y}^{ph}_{pp}))(1 + \escal{P}_3r^2 + \escal{P}_4r^4 ...) \\
\escal{y}^{u} =& (\escal{y}^{ph} - \escal{y}^{ph}_{pp})(1 + \escal{K}_1r^2 + \escal{K}_2r^4 + ...) + \\
&(\escal{P}_2(\escal{r}^2 + 2(\escal{y}^{ph} - \escal{y}^{ph}_{pp})^2) + 2\escal{P}_1(\escal{x}^{ph} - \escal{x}^{pp})(y^{ph} - \escal{y}^{ph}_{pp}))(1 + \escal{P}_3r^2 + \escal{P}_4r^4 ...) \\
\escal{r} =& \sqrt{(\escal{x}^{ph} - \escal{x}^{ph}_{pp})^2 + (\escal{y}^{ph} - \escal{y}^{ph}_{pp})^2}
\end{split}
\end{equation}

where:

\begin{itemize}
  \item $\escal{x}^{u}$ and $\escal{y}^{u}$  are undistorted coordinates also corrected of the principal point.
  \item $\escal{x}^{ph}$ and $\escal{y}^{ph}$ are measured photo coordinates.
  \item $\escal{K}_{i}$ are radial distortion coefficients.
  \item $\escal{P}_{i}$ are tangential distortion coefficients.
\end{itemize}


Relation of undistorted point to object point in camera system can be calculated from 
triangles similarity:

\begin{equation}
\begin{split}
&\escal{x}^{u} =  - \escal{f} \frac{\escal{X}^{cam}}{\escal{Z}^{cam}}   \\
&\escal{y}^{u} = - \escal{f} \frac{\escal{Y}^{cam}}{\escal{Z}^{cam}} 
\end{split}
\end{equation}


\subsubsection{Exterior orientation}

After transformation of 2D photo points into 3D camera coordinate system it is needed to transform it into object space,
because usually the object coordinate system is not coincident with camera coordinate system. 
Object space coordinate system must be cartesian otherwise it must be transformed into cartesian one before applying equations 
derived bellow. 

Exterior orientation is defined by position of camera center in object space and orientation of camera coordinate system in the object space.
Photogrammetry uses representation of orientation by euler angles and rotation matrix.
The euler angles are group of three angles, which describes three subsequent rotations
 around axes of 3D coordinate system. 
 Order of euler angles is important, because in case of breaching it, resulting orientation is different. 
 In this work there is used BLUH notation \cite{baumker2001new} where rotation around y axis by angle $\escal{\phi}$ is 
 perform at first, then system is rotated around x axis  by angle $\escal{\omega}$ and
 the last rotation is perform around z axis by angle $\escal{\kappa}$.
 
Mathematically the three euler rotations can be expressed as multiplication of three corresponding
rotation matrices:
 \begin{equation}
 \begin{split}
\ematr{R} &= \ematr{R}_{z}(\escal{\kappa}) \ematr{R}_{x}(\escal{\omega}) \ematr{R}_{y}(\escal{\phi}) \\
	  &= \begin{pmatrix}
	      cos(\escal{\kappa}) & sin(\escal{\kappa}) & 0 \\
	      -sin(\escal{\kappa}) & cos(\escal{\kappa}) & 0 \\
	      0 & 0 & 1
	      \end{pmatrix}
	      \begin{pmatrix}
	      1 & 0 & 0 \\
	      0 & cos(\escal{\omega}) & sin(\escal{\omega}) \\
	      0 & -sin(\escal{\omega}) & cos(\escal{\omega} 
	      \end{pmatrix}
	      \begin{pmatrix}
	      cos(\escal{\phi}) & 0 & -sin(\escal{\phi}) \\
	      0 & 1 & 0 \\
	      sin(\phi) & 0 & cos(\phi)      
	      \end{pmatrix} \\
	  &=  
	      \small
	      \begin{pmatrix}
	      cos(\escal{\kappa}) cos(\escal{\phi}) + sin(\escal{\kappa}) sin(\escal{\omega}) sin(\escal{\phi}) & 
	      sin(\escal{\kappa}) cos(\escal{\omega})  & 
	      -cos(\escal{\kappa}) sin(\escal{\phi}) + sin(\escal{\kappa}) sin(\escal{\omega}) cos(\escal{\phi}) 
	      \\
	      -sin(\escal{\kappa}) cos(\escal{\phi}) + sin(\escal{\kappa}) sin(\escal{\omega}) sin(\escal{\phi}) & 
	      cos(\escal{\kappa}) cos(\escal{\omega})  & 
	      sin(\escal{\kappa}) sin(\escal{\phi}) + cos(\escal{\kappa}) sin(\escal{\omega}) cos(\escal{\phi}) 
	      \\	      
	      cos(\escal{\omega}) sin(\escal{\phi}) & 
	      -sin(\escal{\omega})  & 
	      cos(\escal{\omega}) cos(\escal{\phi}) 
	      \\		      
	      \end{pmatrix} 
\end{split}
\end{equation}

BLUH angles can be retrieved from rotation matrix using these equitations: 
 \begin{equation}
 \begin{split}
\escal{\phi} &=  arctan \left(\frac{\ematr{R}_{31}}{\ematr{R}_{33}} \right)\\
\escal{\omega} &= arctan \left(\frac{-\ematr{R}_{32}}{ \sqrt{ \ematr{R}_{12}^{2} + \ematr{R}_{22}^{2} }} \right)\\
\escal{\kappa} &=  arctan \left(\frac{\ematr{C}_{12}}{\ematr{R}_{22}} \right)
\end{split}
\end{equation}

Arkus tangens should be calculated with function, which puts final angle into proper quadrant, according to sings of numerator 
and denominator. Usually this functions is called \term{atan2}. 

Pitfall of euler angles is when value of second euler angle is 90 or 270 degrees,  
then euler angles representation is singular, which is called gimbal lock. 
If values  of euler angles are close to the gimbal lock, it could cause series numerical instabilities.

\subsubsection{Basic formula of photogrammetry}

Merging interior and exterior orientation together, direct relation between corresponding 3D point in object space and 2D point
in photo:

\begin{equation}
\label{eq:col_eqs}
\begin{split}
&\escal{x}^{ph} = \escal{x}^{pp} -\escal{f}\frac{\ematr{R}_{11}(\evect{X}^{obj} - \evect{X}^{obj}_{pc}) + 
                                  \ematr{R}_{12}(\evect{Y}^{obj} - \evect{Y}^{obj}_{pc}) + 
                                  \ematr{R}_{13}(\evect{Z}^{obj} - \evect{Z}^{obj}_{pc})                                  
                                  }{
				  \ematr{R}_{31}(\evect{X}^{obj} - \evect{X}^{obj}_{pc}) + 
                                  \ematr{R}_{32}(\evect{Y}^{obj} - \evect{Y}^{obj}_{pc}) + 
                                  \ematr{R}_{33}(\evect{Z}^{obj} - \evect{Z}^{obj}_{pc})     
                                  } +  \\
&\escal{y}^{ph} = \escal{y}^{pp} -\escal{f}\frac{\ematr{R}_{21}(\evect{X}^{obj} - \evect{X}^{obj}_{pc}) + 
                                  \ematr{R}_{22}(\evect{Y}^{obj} - ) + 
                                  \ematr{R}_{23}(\evect{Z}^{obj} - \evect{Z}^{obj}_{pc})                                  
                                  }{
				  \ematr{R}_{31}(\evect{X}^{obj} - \evect{X}^{obj}_{pc}) + 
                                  \ematr{R}_{32}(\evect{Y}^{obj} - \evect{Y}^{obj}_{pc}) + 
                                  \ematr{R}_{33}(\evect{Z}^{obj} - \evect{Z}^{obj}_{pc})     
                                  }
\end{split}
\end{equation}

These two equations are called collinearity equations, where
\begin{itemize}
  \item $\escal{x}^{ph}$ and $\escal{y}^{ph}$ are photo points (equation are expressed in simple form without taking distortion into account)
  \item $\evect{X}^{obj}$, $\evect{Y}^{obj}$, $\evect{Z}^{obj}$ are the point coordinates in object space
  \item $\ematr{R}$ is rotation matrix describing camera system orientation
  \item $\evect{X}^{obj}_{pc}$, $\evect{Y}^{obj}_{pc}$, $\evect{Z}^{obj}_{pc}$ are coordinates of projective center
  \item \escal{f} is focal length
\end{itemize}

\subsection{Short introduction into computer vision}

\subsubsection{Homogeneous coordinates}

Computer vision uses homogeneous coordinates, which allows to mathematically express some relations
 in more elegant way than using coordinates in more common cartesian form. 

3D point in homogeneous coordinates is defined as:

\begin{equation}
\ehvect{x} = (\escal{x}, \escal{y}, \escal{z}, \escal{w})
\end{equation}

\escal{w} component is called scale.

The transformation of homogeneous point into cartesian coordinates is done dividing 
the coordinates by scale:
\begin{equation}
\evect{x} = (\escal{x} / \escal{w}, \escal{y} / \escal{w}, \escal{z} / \escal{w})
\end{equation}

Using homogeneous coordinates it is possible express all cartesian points plus points in infinity.
Infinite homogeneous point is written as: 

\begin{equation}
\ehvect{x} = (\escal{x}, \escal{y}, \escal{z}, \escal{0})
\end{equation}

which is impossible to express using cartesion coordiantes with finite numbers:

\begin{equation}
\evect{x} = (\escal{x} / \escal{0}, \escal{y} / \escal{0}, \escal{z} / \escal{0})
\end{equation}

TODO graphic interpretation of hom coords 

Picture shows 


Homogeneous coordinates form allows to 
use simple vector linear algebra operations 
to get useful information about plane in \eucl{3} and line in \eucl{2}.

Plane in \eucl{3} can be written as:

\begin{equation}
\escal{a}\escal{x} + \escal{b}\escal{y} + \escal{c}\escal{z} + \escal{d} = 0
\end{equation}

Using homogeneous coordinates it can be expressed as vector:

\begin{equation}
\ehvect{\pi} =  (\escal{a}, \escal{b}, \escal{c}, \escal{d})
\end{equation}


Similarly a line in \eucl{2} is defined by equation:
\begin{equation}
\escal{a}\escal{x} + \escal{b}\escal{y} + \escal{c} = 0
\end{equation}

Using homogeneous coordinates it can be also written as vector:

\begin{equation}
\ehvect{l} =  (\escal{a}, \escal{b}, \escal{c})
\end{equation}

It is easy to find out whether point lies on line in \eucl{2}:

\begin{equation}
\ehvect{x}^{T} \ehvect{l} = 0
\end{equation}

or whether a point lies on plane \eucl{3}:
\begin{equation}
\ehvect{x}^{T} \ehvect{\pi} = 0
\end{equation}


In \eucl{3}, a line cannot be written so elegantly.
One of options is to define it given points $\ehvect{p}$ and $\ehvect{q}$ as:
\begin{equation}
 \ehvect{l} = \escal{\mu}\ehvect{q} + \escal{\lambda}\ehvect{p}
\end{equation}

If the point is given as direction of the line $\ehvect{d} = (x, y, z, 0)$ (it is point in infinity), the equation get simplified to:
\begin{equation}
\label{eq:hline}
\ehvect{l} = \ehvect{d} + \escal{\lambda}\ehvect{p}
\end{equation}

Computation of lines ($\ehvect{l}, \ehvect{l}'$) intersection  can be done also in very simple way by cross product in \eucl{2}:

\begin{equation}
\ehvect{x} = \ehvect{l} \times \ehvect{l}' \label{eq:def_hline_pts}
\end{equation}

Unlike cartesian coordinates, homogeneous form allows to express intersection of parallel lines, which lies at infinity. 
In this case the point lies in infinity thus it's scale is equal to 0.

Similarly line can be derived as cross product of two points:

\begin{equation}
\ehvect{l} = \ehvect{x} \times \ehvect{x}'
\end{equation}


\subsubsection{Basic formula of computer vision}

Computer vision uses this equation to describe relationship between
object point and photo point:

\begin{equation}
\begin{pmatrix}
   \escal{x} \\
   \escal{y} \\
   \escal{1} \\
\end{pmatrix}
=
\begin{pmatrix}
   & \escal{f_{x}} & 0     & \escal{c_{x}}\\
   & 0     & \escal{f_{x}} & \escal{c_{x}}\\
   & 0     & 0     & 1\\
\end{pmatrix}
\begin{pmatrix}
&\ematr{R}&\evect{t}\\
\end{pmatrix}
\begin{pmatrix}
   \escal{X} \\
   \escal{Y} \\
   \escal{Z} \\
   \escal{1} \\
\end{pmatrix}
\end{equation}

In essence the equation is equivalent to collinearity equations \eqref{eq:col_eqs} which are used in photogrammetry.
The two models differs only in definition of camera coordinate system.
Unlike photogrammetric camera system (Figure \ref{fig:ph_model}), computer vision defines z axis in the opposite direction heading into scene \cite[p. 156]{Hartley2004}.
Both systems are right handed.
Otherwise the basic formula embodies exactly same transformations as collinearity equations written in different way.

\begin{figure}[h]
    \centering
    \includegraphics[scale=0.3]{figures/photogrammetric_model.png}
    \caption{Computer vision camera coordinate system in arbitrary object coordinate system.}
    \label{fig:cv_model}
\end{figure}

The equation can be abbreviated as:

\begin{equation}
\label{eq:p_exp}
\ehvect{x} = \ematr{K} \ematr{[}\ematr{R}|\evect{t}\ematr{]} \ehvect{X}
\end{equation}

where:

\begin{itemize}
\item \ematr{K} is calibration matrix which contains interior orientation parameters.  
\item \ematr{[\ematr{R}|\evect{t}]} matrix contains rotation matrix\ematr{R} and translation vector $\evect{t}$, 
	      which transforms point from object coordinate system 
	      into camera coordinate system. 
	      The vector $\evect{t}$ can  be transformed by:
	      \begin{equation}
	      \ehvect{C} = - \ematr{R}^{T}\evect{t}
	      \end{equation}
	      giving coordinates of camera projective center. The rotation matrix \ematr{R} and projective 
	      center object coordinates $\evect{C}$
	      represents exterior orientation of camera.   
\item $\ehvect{X}$ euclidean object space coordinates extended into homogeneous form
\item $\ehvect{x}$ photo space homogeneous coordinates extended into homogeneous form
\end{itemize}

Whole projection can be merged into single matrix \ematr{P}, which is called camera matrix:

\begin{equation}
\label{eq:p_abbr}
\ehvect{x} =  \ematr{P} \ehvect{X} = \ematr{K} \ematr{[}\ematr{R}|\evect{t}\ematr{]} \ehvect{X}
\end{equation}

Camera matrix \ematr{P} defines reprojection up to scale $\escal{\lambda}$, thus every camera matrix multiplied by scale gives same photo points:

\begin{equation}
\ehvect{x}=\lambda\ehvect{P}\ehvect{X}
 \end{equation}

The equation can be interpreted as representation of ray going through the photo point and projective center. 
The object point can be everywhere on this ray (depending on $\lambda$ parameter value), getting always 
same photo point coordinates.

\subsubsection{Two photos geometry}

In this chapter, there is introduced theory, which describes mutual relation of two photos. 
The core of the relation arises from epipolar geometry.

\begin{figure}[h]
    \centering
    \includegraphics[scale=0.3]{figures/epipolar.png}
    \caption{Epipolar geometry.}
    \label{fig:epipolae}
\end{figure}

If same object point is identified on two photos then there exists epipolar plane $\pi$.
The epipolar plane contains:

\begin{itemize}
\item Two rays which connects object point, with photo points and camera centers of both photos.
\item Baseline \escal{b}, which connects projection centers of images.
\item Epipoles (\escal{e}, \escal{e'}). Epipole lies on intersection of poto plane with baseline.
\item Epipolar lines  $(\escal{l}_{e}, \escal{l'}_{e})$. Epipolar line is in photo plane going through epipole and photo point.
\end{itemize}


Other relations, which comes form epipolar geomatry are:

\begin{itemize}
\item All epipolar lines of  photo intersect in epipole. If images are merely translate whith no diffrence in orientaion, 
the epipoles are in infinity.
\item Photo point forms epipolar line in the other photo. It allows to reduce space of possible occurrence of the point 
      from \eucl{2}, \eucl{1} of epipolar line, without any other information about the point e. g. object space coordinates. 
\end{itemize}

Rest of this section deals with algebraic deriving of most important relations of two photos geometry. 

Camera matrix \ematr{P} represents projection from object system to first photo system.
Reverse the transformation can not be done with inverse matrix because of the rectangular shape (3x4) of 
camera matrix \ematr{P}. Instead of matrix inversion,  pseudo inverse matrix $\ematr{P}^{+}$ must be 
calculated to reverse the transformation. Pseudo inverse matrix has this property: $\ematr{P}\ematr{P}^{+} = \ematr{I}$.
Transformed photo point

\begin{equation}
\ehvect{x'} =  \ematr{P'}\ematr{P}^{+}\ehvect{x}
\end{equation}

is point in infinity, which represents direction of ray.
If this point is projected into second photo by camera matrix $\ematr{P'}$ resulting point in infinity 
represents direction of epipolar line $\escal{l'}_{e}$. One variant, how a line can be defined, is by direction 
and point. In order to define epipolar line $\escal{l'}_{e}$ missing point can be obtained from 
projection of first camera projective center $\evect{C}$ into the second photo, where it represent
epipole:

\begin{equation}
\ehvect{e'} =  \ematr{P'}\ehvect{C} \\
\end{equation}

%According to  \eqref{eq:hline} line can be given as:

%\begin{equation}
%\ematr{X}(\escal{\lambda}) = \ematr{P}^{+}\ehvect{x} + \escal{\lambda}\ehvect{C}
%\end{equation}

Epipolar line is given using \eqref{eq:def_hline_pts} by:

\begin{equation}
\ehvect{l'_{e}} =  \ehvect{e'} \times \ehvect{x'} = (\ematr{P'}\ehvect{C}) \times \ematr{P'}\ematr{P}^{+}\ehvect{x}
\end{equation}

Cross product of two vectors \evect{a} and \evect{b} can be likewise written as multiplication of matrix and vector:

\begin{equation}
\evect{a}  \times \evect{b}  = 
\begin{pmatrix}
   & 0      & \escal{a_{3}}   & \escal{a_{2}}\\
   & \escal{a_{3}}  & 0               & \escal{-a_{1}}\\
   & \escal{-a_{2}} & \escal{a_{1}}   & 0\\
\end{pmatrix}
\evect{b} = [a]_{\times} b
\end{equation}

Using matrix notation of cross product gives:
\begin{equation}
\ehvect{l'_{e}}  = [\ehvect{e}']_{\times} \ematr{P'}\ematr{P}^{+}\ehvect{x}
\end{equation}

It is apparent from this equation, that matrix, which transforms photo points in the first photo into 
epipolar lines of the other photo and hence denoting relation of two images is defined as:

\begin{equation}
\ematr{F}  = [\ehvect{e}']_{\times} \ematr{P'}\ematr{P}^{+}
\end{equation}

Matrix \ematr{F} is called fundamental matrix. 

Fundamental matrix can be expanded by \eqref{eq:p_abbr}, which
allows to derive another important matrix of two view geometry, which is called essential matrix \ematr{E}.

Lets suppose that camera matrix of first photo is defined in this way:
\begin{equation}
\label{eq:rel_or_cam1}
\ematr{P}  = \ematr{K} \ematr{[}I|0\ematr{]}
\end{equation}
It means that camera coordinate system coincides with object space coordinate 
system.

It's pseudo inverse matrix is:
\begin{equation}
\ematr{P}^{+} =
\begin{pmatrix}
   \ematr{K}^{-1} \\
   \evect{0^{T}} \\
\end{pmatrix}
\end{equation}

homogeneous projection center coordinates are:
\begin{equation}
\ehvect{C} =
\begin{pmatrix}
   \evect{0} \\
    1 \\
\end{pmatrix}
\end{equation}

The second camera matrix is defined in this way:
\begin{equation}
\label{eq:rel_or_cam2}
\ematr{P}'  = \ematr{K} \ematr{[}\ematr{R}|\evect{t}\ematr{]}
\end{equation}


Epipole is given as:
\begin{equation}
\begin{split}
\label{eg:second_epipole}
\ehvect{e'} &=  \ematr{P'}
\begin{pmatrix}
   \evect{0} \\
    1 \\
\end{pmatrix} \\
&= \ematr{K}' \evect{t}\\
\end{split}
\end{equation}


This relationship of two cameras is also called \term{relative orientation}, because rotation matrix \ematr{R} and vector \evect{t}
denotes position and orientation of the second camera to the first one. In order to express cameras exterior orientations in world coordinate 
system or merge it with another relative orientations it is needed to perform further transformations of the relative 
exterior orientations, for more information see chapters \ref{sec:ess_chain} and \ref{sec:helmert}.

Taking into account the relative orientation the fundamental matrix equation can be rewritten as:
\begin{equation}
\label{eg:funeq}
\begin{split}
\ematr{F}  &= [\ematr{P}'\ehvect{C}]_{\times} \ematr{P}'\ematr{P}^{+}
= [\ematr{K}' [\ematr{R}|\evect{t}]
\begin{pmatrix}
   \evect{0} \\
    1 \\
\end{pmatrix}]
_{\times} 
\ematr{K}' [\ematr{R}|\evect{t}]  
\begin{pmatrix}
   \evect{K}^{-1} \\
   \evect{0}^{T} \\
\end{pmatrix} \\
&= [\ematr{K}' \evect{t}]_{\times} \ematr{K}'\ematr{R}\ematr{K}^{-1} \\
&= [\ehvect{e}']_{\times} \ematr{K}'\ematr{R}\ematr{K}^{-1} 
\end{split}
\end{equation}

Now there comes little bit more difficult part.
This formula applies for any non-singular matrix \ematr{K} and vector \evect{t}:
\begin{equation}
[\evect{e}']_{\times} \ematr{K'} = \ematr{K}'^{-T}[\ematr{K}'^{-1}\evect{e}']_{\times}
\end{equation}

Using the formula to \eqref{eg:funeq}, it is given:  
\begin{equation}
\begin{split}
\ematr{F}  &= [\ehvect{e}']_{\times} \ematr{K}' \ematr{R} \ematr{K}^{-1} \\
	   &= \ematr{K}'^{-T} [\ematr{K}'^{-1} \ematr{K}' \evect{t}]_{\times} \ematr{R} \ematr{K}^{-1} \\
	   &= \ematr{K}'^{-T} [\evect{t}]_{\times} \ematr{R} \ematr{K}^{-1}
\end{split}
\end{equation}

This equation reveals very important fact about fundamental matrix, which says that the transformation performed 
by matrix can be divided into three main steps.
At first photo point of first image is transformed into the first camera coordinate system. Result of the transformation is 
point in the infinity, which describes ray represented by photo point. 
After that the ray is projected into second camera coordinate system. This is done by so called essential matrix:
\begin{equation}
	 \ematr{E}  = [\evect{t}]_{\times} \ematr{R}
\end{equation}


As the last step, the ray is projected into the second photo as epipolar line.


It implies that with known exterior orientation it is possible to define essential matrix. 
Moreover if interior orientation of cameras of both photos are known it is possible to determine fundamental matrix,
which describes relation of a two photos.

\subsubsection{Triangulation of points}
\label{eq:triang}

If measurements would be precise, object point 
would lie on intersection of two rays which is clearly visible on 
 Figure \ref{fig:obj_intersec}. Unfortunately the rays are never determined so precisely to intersect.
Triangulation methods deals with the imprecision.  

Direct linear transformation (DLT) method  is based 
on lineralization of collinearity equations \eqref{eq:col_eqs}
employing least square method on it.

The linearized collinearity equations can be written as:  
\begin{equation}
\label{eq:col_eqs}
\begin{split}
&\escal{x}^{un} = \frac{\escal{a}_{1}(\evect{X}^{obj}) + 
                                  \escal{a}_{2}(\evect{Y}^{obj}) + 
                                  \escal{a}_{3}(\evect{Z}^{obj}) +
                                  \escal{a}_{4}
                                  }{
				  \escal{a}_{9}(\evect{X}^{obj}) + 
                                  \escal{a}_{10}(\evect{Y}^{obj}) + 
                                  \escal{a}_{11}(\evect{Z}^{obj}) +
                                   1  
                                  } \\
&\escal{y}^{un} = \frac{\escal{a}_{5}(\evect{X}^{obj}) + 
                                  \escal{a}_{6}(\evect{Y}^{obj}) + 
                                  \escal{a}_{7}(\evect{Z}^{obj}) +                                 
                                  \escal{a}_{8}
                                  }{
				  \escal{a}_{9}(\evect{X}^{obj}) + 
                                  \escal{a}_{10}(\evect{Y}^{obj}) + 
                                  \escal{a}_{11}(\evect{Z}^{obj}) +    
                                  1
                                  }
\end{split}
\end{equation}
 
It is easy to implement and 
there is no need to know approximate values. 
Main drawback of this method is that DLT model does not respects pinhole camera model, thus
minimized parameters in vector \evect{a} do not follow rigorous camera geometry which 
may result in inaccurate retrieval of object coordinates.

The more complicated method is called the optimal solution. 
It is based on enforcing epipolar constraint.  Cost function 
minimizes sum of distance of photo points to epipolar lines, which are created 
by projecting corresponding photo points form other images.

The cost function is not linear however it's derivative leads into 6 degree polynomial thus 
it has 6 roots. The root, which represents correct solution gives smallest values of cost functions, 
because it represents global minimum. 

This method is considered by computer vision community as most accurate \cite[p. 315]{Hartley2004}.  

\subsubsection{Retrieving of relative orientation from essential matrix}
\label{sec:ess_eo}


Rotation \ematr{R} and vector \evect{t} can be retrieved from essential matrix using 
singular values decomposition (SVD):

\begin{equation}
\begin{split}
\ematr{W} &=
\begin{pmatrix}
& 0 & -1 & 0 \\
& 1 & 0 & 0 \\
& 0 & 0 & 1\\
\end{pmatrix}] \\
+- \ematr{R} &= \ematr{U}  \ematr{W} \ematr{V}^{T}\\
+- \evect{t} &= \evect{U}_{3,:}
\end{split}
\end{equation}

where matrices \ematr{U} and \ematr{V} are two rotation matrices  obtained for SVD decomposition.
The retrieved rotation \ematr{R} matrix and the shift vector \evect{t} are defined up to sign,
which causes four case ambiguity. 

As a consequence of the  ambiguity it gives four relative orientations: 
\begin{equation}
[\ematr{R}|\evect{t}],
[\ematr{R}|\evect{-t}],
[\ematr{-R}|\evect{t}],
[\ematr{-R}|\evect{-t}],
\end{equation}


The correct solution can be found   by assumption that all object points has to be in front
of both photo planes.

\begin{figure}[h]
    \centering
    \includegraphics[scale=0.3]{figures/eo_ambiguity.png}
    \caption{Four possible relative orientations of cameras.}
    \label{fig:rel_or_amb}
\end{figure}


\subsubsection{Merging of relative orientations}
\label{sec:ess_chain}

If two relative orientations are merged together, as first step, it is necessary to calculate scale of the one orientation
to the other because the systems are defined up to scale.
The scale can be calculated from corresponding distances in both coordinate system. 
It can be used distances of 
object points and object coordinates of projective center of camera, which are present in both relative orientations.

The scale can be computed by this equation:
\begin{equation}
\escal{s}_{ro2-1} = \frac{\mathbf{\evect{X}^{ro1}_{1} - \evect{C^{ro1}_{1}}}}
	                {\mathbf{\evect{X}^{ro2}_{1} - \evect{C^{ro2}_{1}}}}
\end{equation}

where \term{ro1} denotes destination relative orientation of transformation and \term{ro2} is the transformed relative orientation. 

The equation gets simplified if the relative orientation coordinate system  corresponds to the shared camera system: 
\begin{equation}
\escal{s}^{ro2-1} = \frac{\mathbf(\evect{X}_{ro1})}
	                 {\mathbf(\evect{X}_{ro2})}
\end{equation}

Commonly there are processed bundles of photos capturing a scene. Relative orientation pairs of photos capture only fragments of the scene, 
therefore the information of scene is scattered.  
In order to interpret the scene as whole it is needed to merge all relative orientation systems into \term{common relative coordinate system}.
For sake of simplicity, the common relative system is defined by arbitrary relative orientation of pair and other 
relative orientations are subsequently transformed into the common system. 

Group of photos, which can be transformed into common system, is composed of the relative orientations pairs
which comprise connected component of undirected graph where nodes represents  photos and edges denotes  relative orientations. 
In other words, it is possible to transform relative orientation into the common relative coordinate system only if there 
exist continuous chain of relative orientations where every two neighborhood relative orientations in the chain shares one photo.

The transformation of relative orientations from 1 to n, with common system defined by relative orientation 1 can be written as: 
\begin{equation}
\label{eq:comm_rel}
\begin{split}
&\ematr{P}_{1} = [\ematr{I}|\evect{O}] \\
&\ematr{P}_{2} = [\ematr{R}^{ro1}|\evect{t}^{ro1}] \\
&\ematr{P}_{3} = [\ematr{R}^{ro1} \ematr{R}^{ro2}| \ematr{R}^{ro1} \evect{t}^{ro2} \escal{s}^{ro2-1}] \\
&\ematr{P}_{n} = [\ematr{R}^{ron} \ematr{R}^{ro(n - 1)}| R^{ro(n - 1)} \evect{t}^{ron} {s}^{ron- (n-1)}] \\
\end{split}
\end{equation}


The equations applies for chain of relative orientation, where the shared camera in first relative orientation has 
projection matrix defined by $[\ematr{R}|\evect{t}]$ matrix and the merging second relative orientation is 
defined $[\ematr{I}|\ematr{O}]$.

\subsubsection{Helmert transformation}
\label{sec:helmert}

Helmert transformation equation can be written as:
\begin{equation}
\ematr{X}_{t} = \evect{t} + \escal{s}\ematr{R}\evect{X} \\
\end{equation}
where \ematr{R} is rotation matrix, \escal{s} is scale scalar and  \evect{t} is translation vector.
The transformation is defined by seven parameters (three translation coordinates, three rotation angles and scale).
In order to get these parameters which defines relation of two coordinate system it is needed to know at least two corresponding points and  another one with 
at least one known coordinate in both systems. 

Helmert transformation parameters can be retrieved from corresponding points by several methods. It can be determined
iterativaly by non-linear least square methods. There exist also closed form method, which is based on SVD decomposition
\cite{sjoberg2013closed}.


\subsection{Bundle block adjustment}

Bundle block adjustment (BBA) is method, which uses the least squares technique (\ref{sec:least}) to refine scene parameters. 
The main advantage of BBA is that the method takes into account whole scene and therefore it utilizes 
maximum information for adjustment of scene parameters, which leads to more accurate results.
The principle of BBA has been known for very long time since 1950s,
however the method has been employed in practical applications since 1990's, because the performance of computers had not been 
sufficient enough before. 


\begin{figure}[h]
    \centering
    \includegraphics[scale=0.3]{figures/bba.png}
    \caption{BBA takes into account scene as whole.}
    \label{fig:rel_or_amb}
\end{figure}

Cost functions of BBA least squares method are photogrammetric collinearity equations \eqref{eq:col_eqs}.
The least squares method tries to find such a vector of adjusted parameters \evect{X} to minimize
sum  of square differences between measured photo points coordinates and photo points coordinates calculated 
from collinearity equations using adjusted parameters \evect{X}.

Unfortunately collinearity equations are non linear, hence the non-linear least square method \ref{sec:non_least} must be employed, which 
needs to be provided  whith sufficiently accurate initial values in order to find values of parameters which satisfies minimization criterion.

Thanks to the least squares method BBA is very flexible in terms of selection of parameters to be adjusted. 
It is possible to select various combinations of interior, exterior orientations and object points coordinates parameters. 

Following example describes BBA in practice.
Lets suppose that inital values of adjusted parameters are known with sufficient accuracy in order to the least squares
method eventually iterate to global minimu.
Lets say that there is a scene, which is captured by three photos.
There are identified three tie points and two ground control points in all three photos.
Tie point is kind of point, which is identified only on photos with missing information about its object coordinates.  
Unlike tie point, object coordinates of ground control point (GCP) are known.
Additional two tie points and one GCP are identified in two photos.
Also another one ground control point and one tie point are identified 
in only one photo. 

%Initial values of tie point object coordiantes were obtained, but the values needed to be refined by BBA to improve its accuracy.

Every tie point needs to refine three parameters (object coordinates). 
Object coordinates of GCPs are known with high accuracy, therefore they are not refined by BBA. 
If the tie point is identified on a photo it yields two measurements (x, y collinearity equations), 
therefore it is needed to identify the tie point at least on two photos to get more measurements 
than unknowns. This makes perfectly sense because it is not possible to determine object point coordinates from one photo,
because it can lie wherever on the ray (See Figure \ref{fig:obj_intersec}).

Unlike tie points, GCPs does not bring any new adjusted parameters, which 
allows to use the GCP visible only in single photo. In this case, the object coordinates 
are known, thus there is no ray ambiguity. 

Lets suppose that it is adjusted exterior orientations and object coordinates of tie points.  
Every photo yields six parameters of exterior orientation (three projective center coordinates and three orientation angles) 
which gives 6 * 3 = 18 parameters. 

Tie points gives 5 * 3 = 15 adjusted parameters of object coordiantes. 
Only 5 tie points are used in BBA because the tie point visible in just one photo is skipped because of the ray ambiguity.
GCPs does not gives any adjusted parameters since coordinates are known.

Every tie point and GCP, visible on three photos gives 6 measurements (two collinearity equations for every photo point),
which gives totally 30 measurement.
Points visible from two photos gives 4 measurements, which totally yields 12 measurements.
GCP which is visible from single photo gives additional 2 measurements.

Overall score is 33 adjusted parameters and 44 measurements.

Thanks to 9 redundant measurements it is possible to add some of the interior parameters into adjustment.  
On the other hand the redundancy of measurement improves accuracy of results, which is the main point of the least square method,
therefore it should be avoided to have nearly same number of measurements and adjusted parameters.

In first least squares iteration, initial values of adjusted parameters are put into parameters vector $\evect{X}$. 
Vector $\evect{L}_{X0}$ can be computed applying the parameter vector $\evect{X}$ on collinearity
equations. In order to calculate all collinearity equations and its derivatives it is also needed to 
use values of not adjusted parameters which are not presented in the parameters vector.
Measured photo points are represented by vector $\evect{L}$.
Design matrix $\ematr{A}$ is comprised of values of  collinearity equations partial derivations respect to adjusted parameters
given by parameters vector $\evect{X}$. Columns of the design matrix represents partial derivatives 
respects to corresponding parameter  in  $\evect{X}$  (mumber of column equals to position of parameter in  $\evect{X}$ ).
Analog logic applies for design matrix row and position in $\evect{L}$ or $\evect{L}_{X0}$ vectors.
In the example, design matrix has 44 rows and 33 columns $\ematr{A}$.


%Let say that first two rows (x, y) in design matrix represents some photo point of GCP.
%It follows that first two positions in vector  $\evect{L}$ represents measured x, y photo coordiantes.
%First two values of vector $\evect{L}_{X0}$ are computed  


All needed information is available to perform first iteration of non linear least square 
method \label{sec:non_least}. As result of iteration refined values of vector  $\evect{X}$
are obtained. If another iteration is needed to performed same process is repeated 
with adjusted vector $\evect{X}$.

\subsection{Orthorectification}

The orthorectificaion is process, which transforms photo from perspective projection into orthogonal projection (See Figure \ref{fig:ortho}). 
\label{sec:single_ortho}
There are two main methods for producing of orthophoto. The first method is able to retrieve orthophoto 
from single photo. It requires exterior, interior orientations of the photo and shape of relief. 
Shape of relief is usually provided in form of digital terrain model (DTM), which 
can be given in raster or vector form. Raster DTM describes relief with same density defined by resolution of the raster.
Vector DTM is comprised by continuous surface made of triangular irregular network (TIN). 
At the beginning of the orthorectificaiton empty raster is allocated, which represents orthophoto.
It covers intersection of photo scene and area of the DTM.

In the next step, empty orthophoto is filled by values. 
At first all points representing centers of pixels are given Z coordinates from DTM. 
Then the points are transformed into the photo coordinate system using collinearity equations.
Because the transformed point is represented by floating point, value which is given 
to the pixel in orthophoto is deteremined by interpolation of surrounding pixels of the transformed 
point into the photo. There can be many variation in used interpolation methods, selection of points 
for transformation etc. 
but main principle is still the same which is based on back projection of object point into 
the photo to obtain values for orthophoto.

The other method can be considered a extension of the first one because initially it generates DTM than it recovers orthophoto 
according to same principle as the first method. The principle of the method is based on triangulation of object point
from photo points, which requires to be known interior and extrerior orientation of photos. The reconstructed point 
must be visible at least on two photos in order to solve the ray ambiguity. It follows that it requires more than 
one photo in contrast to the first method.
Object coordinate system 
can be defined arbitrarily, because there is no needed to define relationship to input DTM as the first method requires.
In stead of this DTM is created in any system directly by the method.
Hence,  orthophoto and DTM can be generated without any GCP. 
Approximate object points  are calculated by triangulation 
methods \ref{sec:triang} and than refined by BBA. 

The DTM is created from retrieved object coordinates. There 
are plenty of methods for DTM creation from points. Raster DTM can be created by various interpolation methods, 
which predicts value of raster according to surrounding points. Commonly used method for  creation of 
vector DTM (TIN) is Delaunay triangulation, which creates surface comprised of triangle network, corners of which are located 
on object points. The triangulation finds such a configuration of triangles, that no object 
point is inside the circumcircle of any triangle.
In case of processing big number of points Finite element method 
should be used instead of Delaunay triangulation, because it is computationally more efficient.
Unlike Delaunay triangulation, which creates surface including all object points, Finite element method
is based on principle of dividing trangle into smaller triangles. 
The triangel is splot to smaller ones until it sufficiently aproximates surrounding object points. 
This method is able to create DTM from millions of points.

\begin{figure}[h]
    \centering
    \includegraphics[scale=0.2]{figures/finite_elements.png}
    \caption{TIN created by Finite element method.}
    \label{fig:rel_or_amb}
\end{figure}

After obtaining of DTM, orthophoto can be created using same principle, which was described in the first method. 

\section{Analytical part}

\subsection{General solution of UAV bundle block adjustment}


In 1990s Global Positioning System (GPS) become first globally operational satellite navigation.
GPS allows 
to locate position of GPS unit with accuracy ranging  from  millimeters to decimeters depending 
on used technique and other conditions. 
GPS greatly simplified obtaining of initial values of exterior orientation in photogrammetric aerial mapping missions.
GPS unit together with 
inertial measurement unit (IMU), which measures its orientations in object space,
provides all needed parameters of exterior orientation with sufficient accuracy. In order to be information useful it is need to determine 
leverage arm offset, which describes shift of GPS unit to the perspective center and bore sight offset
which describes orientation of camera coordinate system towards IMU coordinate system. These two
quantities are determined in calibration process. The measured values from GPS and IMU units has to
be corrected by the two quantities to represent true exterior orientation of camera coordinate system. The corrected values 
can be refined by BBA if Collinearity equations are extended to include the corrections.

Before GPS was introduced, aerial photogrammetry made some a priori assumptions about the camera position,
which simplified the collinearity equation in order to get initial values of exterior orientation. 
For instance it can be assumed that the aerial photo is vertical if orientation deviates from true vertical
direction up to 2-3 degrees. Using this assumption collinearity equations can be linearized and 
approximate values can be computed easily from GCPs.

Photogrammetric aerial mapping missions are prepared and carried out by experts using 
professional equipment which is properly calibrated.
As a consequence, budget of such a missions
is high. These facts prevents further spreading of rigorous aerial photogrammetry methods.

Nowadays nearly everybody has camera in his pocket, thus there is big demand 
to provide solution which is very simple from the user point of view because 
vast majority of users has limited or no knowledge of photogrammetry or computer vision. 
Together with spread of UAV, the mapping missions can be performed with much lower 
cost compared to rigorous aerial photogrametry mission, which used to employ manned aircraft.

It follows the main requirement 
on solution, which needs to be simple and cheap enough,
in order to be adopted by
ordinary users, who do not want 
to spend money for other devices or spend time studying complicated methods. 
The users just 
want to take a group of photos of the scene and get the job done by the software.
The final solution should be flexible to process both terrestrial and aerial photos
giving both orhophoto and structure / DTM as output. 
No limitation on camera orientation should be enforced, because it should be possible 
to process photos acquired from camera held by hand or mounted on UAV.
Because of instability of UAVs flight track and human hand, the photos 
must be considered as oblique. 

In order to meet the requirement of simplicity, user should just need  
 camera to obtain structure of the scene. Orhophoto creation would also need to employ of UAV
in order to get images from approximately vertical point of view.
No other hardware device should be needed. 
All other steps should be done 
using free software, which can everybody freely download and use. 

The solution should be flexible enough to process additional data, which may be known about a scene, 
e. g. GCPs, positions of cameras etc. The additional data can
 increase accuracy of results and computational time  
 decreasing number of  BBA iterations needed for finding global minimum. This is very important 
 for professional photogrammetric measurements, where accuracy is very important.

As consequence of these constraints, the minimum required input can be just photos
of the scene and other information, which can be derived from photos of calibration scene.

The another important requirement is to make the process as much automatic as possible, because 
if it would require a lot of time of user to go through the processing chain nobody would use it.

This processing chain should fulfill above mentioned requirements for the solution:

\begin{itemize}
\item Camera calibration - determination of camera interior orientation. There exists simple techniques for calibration
of camera based on processing of 2D patterns photos. Whole process is very simple from the user point of view, it is needed 
to print the pattern on sheet of paper, take photos of the pattern from different positions. The calibration 
is done automatically based on the photos. No other user input except  capturing of the photos of pattern is needed.
\item Tie point identification - it can be done fully automatically by pattern matching algorithms.
\item Retrieval of initial values for BBA - it is fully automatic process without need of any manual user input.
As input it needs tie points and interior orientation of camera, which are given by previous two steps.
\item Bundle block adjustment - it is fully automatic integrating information from all three previous steps.
\end{itemize}

 
\subsubsection{Camera calibration}

Camera calibration is process, which deals with determination of interior orientation parameters including distortion.
There exists many different methods of camera calibration.

One family of calibration methods are based on photos capturing calibration object.
The calibration object includes calibration points, which relative position to each other is  precisely determined.  

The methods can be further divided according to dimension of the object:
\begin{itemize}
\item 3D reference object based calibration - This method allows to reach very good accuracy of determined 
interior orientation parameters. One of common calibration objects is comprised of three or two
perpendicular planes, which are covered by calibration points. Drawback of 3D calibration 
is that accurate construction of the calibration object is complicated.
\item 2D plane based calibration - This calibration method is less accurate compared to 3D calibration. 
The accuracy can be improved by acquiring more photos of the calibration object.
On the other hand calibration object can be printed on a sheet of paper, which is much 
easier compared to construction of 3D calibration object.
\end{itemize}


Another method is self-calibration, which finds values of the interior orientation parameters by BBA.
However at least initial values of focal length and principal point must be known.
Usually focal length and chip size are publicly available by manufacturer. The principal point 
can be computed dividing chip size by two, because it is supposed that it lies close to 
the middle of the photo.
It is more convenient to use 
pixels as units of interior orientation, therefore focal length and principal point coordinates can be 
converted from millimeters units into pixels using photo resolution. 
Interior orientation provided based on manufacturer information should not be interpreted as very accurate, therefore 
it has  to be refined by camera calibration.
Including interior orientation as parameters of BBA
can lead into over-parametrization which may result in inaccurate results.
In case of self-calibration it is recommended to take acquire convergent photos from many different directions and position in order
to avoid correlation of adjusted parameters. 

In order to keep interior orientation parameters stable the camera during should have fixed same zoom level,
because interior orientation parameters are very sensitive  to zoom changes \cite{labe2004geometric}.
Influence of focus is  much smaller then zoom influence and it has affect only in short distances of camera
from to scene.
Also change of the resolution may cause changes of interior orientation parameters, which mostly depends 
on specific camera type. 
Change of other camera features as exposure time or aperture does not have so significant effect on interior
orientation.

\subsubsection{Tie points identification}

From developer point of view, the simplest way of tie points identification is to force user to locate tie points on 
the photos manually. However users would have to waste a lot of time on such a 
dull task. Fortunately there  exist algorithms, which are able to identify tie points on photos
automatically. The state of the art algorithms are SIFT \cite{wiki:SIFT} and SURF \cite{wiki:SURF}.
These algorithms allows to identify tie points on corresponding photos capturing same parts of the scene, without need to include any other information.
Unfortunately both algorithms are patented in USA, therefore their use should be avoided in open source community.

Recently new BRISK \cite{leutenegger2011brisk} and FREAK \cite{alahi2012freak} algorithms
has been developed, which does not have patent restriction. 

Another issue of tie points matching is determination of corresponding photos. 
If photos are taken in strip, it is easy, because the corresponding photos are comprised form neighborhood photos 
in the chain.
If there is no a priori information about photo order, the easiest way is to use brute force 
checking all photo pairs combinations to find correspondence.
However it become unsolvable problem with higher number photos, because 
of high number of combination which needs to be checked. In order to avoid this, visibility map \cite{barazzetti2010extraction} can be build up. 
The visibility map is graph with photos as nodes. The graph edges connects photos if there is
correspondence between them.  It is possible to use information from GPS and IMU units for creation of visibility map.

Another speed up can be done decreasing resolution of photos 
for first search of tie points. First search of photos only identifies correspondence.
In second step tie points extraction is done on full resolution but only with correspondent photos.

\subsubsection{Retrieving initial values for bundle block adjustment}

The next step is to utilize all known information (interior orientation and extracted 
photo coordinates of tie points) to get missing initial values of parameters for BBA. 

Because collinearity equations \ref{eq:col_eqs} are the cost functions of BBA,
extracted photo coordinates represents measurement,
the parameters of interior orientation, thus only information, 
which is missing, are parameters of exterior orientation and object coordinates of 
tie points. 

There exist many methods how to obtain initial values of exterior orientations. The methods 
differ in information which they require for input, computation time and accuracy. 
In order to fulfill the set up requirements a chosen method should be able to retrieve exterior orientation 
using just photo coordinates of tie points and interior orientation of camera. The method 
should be general in terms of capability to process oblique photos and robustness to 
various object points configuration.

One group of methods requires knowledge of object point coordinates to retrieve exterior
orientation. 
The requirement of object coordinates would have negative impact on processing 
chain because, it would require manually identify such a points, thus 
such a methods should be avoided. 

Another group of methods is family of closed form computer vision algorithms,
which are able to determine relative orientation (essential matrix) from photo coordinates 
of the tie points. The algorithms are based on epipolar geometry using essential matrix 
properties to find the solution. 

The closed form algorithms are given the names 
according to required number of identified tie points in two photos.
There exist eight, seven, six and five point algorithms. The five point algorithm seems to perform best in most
of cases \cite{stewenius2006recent} and it is not sensitive on planar points configuration as others.

Only case, when the five point algorithm \cite{nister2004efficient} underperforms 
\cite{bruckner2008experimental} the others is forward motion of camera.
This flaw can be eliminate by  joint usage of eight and five point algorithms.

Main disadvantage of the five point algorithm is that it can produce up to ten solution.
Hence it is needed to asses the solutions and select the correct one. The number of solutions
can be reduced by assumption that all tie points lies in front of both photo 
planes. After sorting out obviously non sense solution,
 the correct solution can by selected by assumption that it gives minimum sum of epipolar distances.
 Epipolar distance is photo point distance
 from epipolar line. The epipolar line is computed from corresponding photo point 
 in another photo using fundamental matrix.
  
Because matching algorithms are not perfect, certain number of the tie points are matched 
incorrectly, thus it is needed to be the relative orientation robust enough to
deal with these outliers.

The RANSAC \cite{wiki:RANSAC} method is commonly used with the family of epipolar algorithms to 
make it more robust for effect of outliers.
Principle of RANSAC is finding such  a solution in group of measurements with outliers, 
which well approximates inliers.  

In case of the epipolar algorithms family, RANSAC selects random tie points in every iteration.
The number of selected tie points is chosen according to minimum number, which the epipolar algorithm requires.
The epipolar algorithm is launched with the selected points as input. 
Quality of returned essential matrix by epipolar algorithm is assessed by 
computing epipolar distances of tie points. There is specified some threshold of the distance, which is used 
for decision whether the measurement is inlier or outlier. The other threshold, which has to be specified 
is minimum percentage of inliers from all tested points. If percentage of inliers is higher than in any previous iteration
and it satisfies the minimum percentage of inliers threshold, then
the essential matrix is set as the best fitting and iteration continues. The iteration terminates after 
reaching maximum number of iterations returning the best fitting essential matrix.

After essential matrices are obtained, all needed information for calculation of initial object point coordiantes
is  available. The object points coordinates can be computed by triangulation methods described in \ref{eq:triang}.


\subsubsection{Common relative coordinate system}

All approximate values, which are needed for BBA collinearity cost functions are available. 
However it is not possible to adjust scene in one bundle, because
object coordinates of tie points and exterior orientations are denoted in  individual object coordinate systems
of relative orientation of the photo pairs.

Therefore it is needed to transform all features from individual relative coordinate systems of photo pairs into 
an common relative coordinate system.
Relative orientation systems are subsequently transformed into the common relative system \ref{sec:ess_chain}. 

\subsubsection{Transformation into world coordinate system}
Difference between world coordinate system and common relative system is that the common relative system 
is defined by arbitrary selected relative orientation. World coordinate system definition is standardized.
Thanks to the standardized definition it is connected to all other commonly used  world systems through 
defined transformations to these systems.
GCP object coordinate are mostly expressed in some world coordinate system, in order to be 
information of precise GCsP coordinates usable in BBA it is needed to transform common relative 
system into the world system of GCPs. Since the scene is transformed into world coordinate system,
it can also be combined with other data which are express in some world coordinate systems e. g.
maps. Moreover availability of precisely determined GCP can significantly improve results of BBA.  

This transformation is called Helmert  transformation which comprises from 
three main steps: rotation, scaling and translation \ref{seq:helmert}.

In order to be transformation of the common system into the world system be possible, it is needed 
to have at least 3 GCPS, with known initial coordinates in the common relative space. 
 
This step is optional, it is possible to perform bundle block adjustment in the common system. In this 
case it is needed to perform BBA of free network \label{sec:free_net_least}. 

\subsubsection{Bundle block adjusmnet}

All initial values of parameters in collinear equations are obtained and transformed into common system therefore 
BBA can be performed.

Main reason for employing BBA is to improve accuracy of results.
In order to obtain initial values it was merely used information from relative orientation of photo pairs.
However many object point are usually visible from more than two photos. This redundancy is not fully taken 
into account during retrieval of initial values. The object points coordinates are calculated 
just from two photos and only redundancy which can appear there is when point is part of more 
photo pairs, which relative orientation was computed. 

Unlike initial values retrieval, BBA utilizes this redundancy rigorously, therefore 
it can significantly improve accuracy of the adjusted parameters.
Commonly initial values of parameters contains certain amount of outliers, caused by instabilities in previous steps of processing chain.
The BBA can also significantly improve accuracy of such a parameters.
Moreover if the point is wrongly identified in a photo, such a point can be recognized and its 
negative effect can be eliminated.

If  there are at least three GCPs and the scene is transformed into world coordinate system the ordinary non-linear least square method \label{sec:non_least} 
can be used for BBA otherwise
it is needed to use free network least squares \label{sec:free_net_least} to cope with datum deficiency. 

\subsection{GRASS GIS Orthorectification Workflow}

GRASS is among few open source GIS software which supports orthorectification
\cite{rocchini2012robust}.Orthorectificaion is done by 
by i.ortho.photo \cite{i.ortho.photo} module.
The module is comprised of submodules, which represents individual steps of 
GRASS orthorectification workflow.


The module was developed in 1990s,  
when analog photos were mostly processed. Therefore it supports 
 transformation of scanned photo using fiducial marks from coordinate system of scanned image into 
 photo coordinate system. The fiducial marks are used for calculation of 
 affine transformation coefficients form known photo coordinates 
 and corresponding scanned image coordinates, which are identified manually. 
 This is done by i.photo.2image submodule, which allows to identify the fiducial 
 marks in scanned image coordinates system and assign known photo coordinates.
  In case of digital photo processing, imaginary fiducial marks has to be defined, which
can be located e. g. in the middle of photo sides.  
  

The interior orientation parameters of camera are supposed to be known and can 
be managed by i.photo.camera submodule.
The data model is not taking distortion 
coefficients into account. 
The next step is retrieval of exterior orientation parameters by i.photo.2target submodule.  
The retrieval 
of exterior orientation is based on non-linear least squares iteration. Unlike BBA,
the photos are adjusted separately, therefore only GCPs can be used,
because it is not possible to determine object coordinates of tie point from one photo.
At least three ground control points has to be available because six exterior parameters are 
searched. The three GCPs are just theoretical minimum, therefore it is needed to provide at least 12 parameters
for getting reasonably accurate results. Exterior orientation 
can be included into input using i.photo.init submodule in order to be used as initial values for least squares iteration,
otherwise it is supposed that images are vertical, therefore two of three rotation
angles are set to 0 and the last angle is computed as difference of a GCPs centroid  slope
in photo system and object system taking into account only x, y coordinates.
Projection center object coordinates are given equal to the centroid coordinates.

In the last step, the orthorectified photo is computed by i.ortho.rectify submodule. GRASS uses single photo orthorectification
method \label{sec:single_ortho}, which requires known DTM in order to deal with height differences. 

The main disadvantaged of GRASS orthorectification method are:
\begin{itemize}
\item It orthorectifies photos separately, thus unlike BBA, it ignores significant part of information,
 which can improve accuracy and robustness of final orthophoto. 
\item Because of single photo orthorectification method it is necessary to attach DTM of scene to the input 
in order to generate orthophoto. Precise high resolution DTM is not usually available for free so this 
requirement can be significant obstacle for producing accurate orthophoto. If sufficiently precise 
DTM is not available, freely available DTM with usually lower resolution can be used, therefore  
accuracy of final orthophoto will be limited by insufficient resolution of DTM.
\item The requirement for at least 3 visible GCPs from every photo can be sometimes also difficult to fulfill.
Close range flight of UAV which can easily acquire hundreds of photos.
In such a case one photo covers tens of meters
of terrain, it is nearly impossible to measure so many ground control points to fulfill the requirement. 
On the other hand this reproach as not valid in context of the workflow, because it requires DTM, therefore 
enough GCPs can be obtained from it.
\item Moreover the workflow does not contain any submodule for retrival of intial exterior orientaion except
 the case of vertical photo.   If the photo 
is not vertical, it is needed to know at least initial exterior orientation 
in order to the least square method converge towards global minimum.
\end{itemize}

These disadvantages are big obstacle which make current implementation of orthorectification difficult to use  
to process UAV or terrestrially acquired photos, which often lack both GCP and DTM information and can not be considered as vertical.
On the other hand the current GRASS orthorectificaiton workflow can be successfully used for processing
of aerial photos from photogrammetric missions.

Currently i.ortho.photo module works in GRASS 6 which includes stable version 6.4.3.
GRASS 7, which is development version, does not fully support all orthorectification features 
as a consequence of change of rendering architecture.
Therefore new front end for the 
orthorectification has to be developed in order to make it fully operational 
in GRASS 7. 

\section{Implementation}

\subsection{Used technologies}

\subsubsection{OpenCV}

OpenCV is open source computer vision and machine learning library released under BSD licence.
The library contains thousands of algorithms, which are mainly focused real time image processing.
Originally it was developed by Intel. Currently OpenCV is supported by non-profit organization OpenCV.org.

The library is written in C++ including interfaces to other programming languages as C, Python, Java etc. 

The OpenCV library contains very useful algorithms which deals with retrieving structure from group 
of photos. It includes algorithms which can determine exterior orientation 
and object coordinates of the tie points.
Recently there was included five point algorithm implementation into development version of OpenCV.
Another relevant group of algorithms deals with camera calibration. OpenCV  camera calibration
is based on calibration of 2D plane based calibration.  
There are also implemented feature matching algorithms, which can be used for tie points identification. 


\subsubsection{Python}

Python is open source programming language which has been developed since 1991. 
The python code is easily readable and therefore it is often picked as first programming
language to learn. Python is employed in wide range software from simple scripts to complex systems.
It is high level programming language compared to languages as C++ or C, therefore 
the development process is usually significantly faster. On the other hand 
Python is significantly (hundred times) slower compared to C++. Despite the big 
slow down Python is usually more preferred over C++/C because 
it is used as top level glue in many application the Python, which take care about 
the main workflow. The parts of application, which do heavy computation, can 
be written in other faster programming languages and then the parts can be called 
from Python. This is very powerful combination, which allows to speed up development 
process preserving speed of application.

Thanks to the its popularity Python world is very rich. There exist
thousands of open source libraries, which make life of programmer much easier. Even 
if library is not implemented in Python it is possible to generate Pythonic 
interface using some of available open source tools (e. g. SWIG), which automatically
generates the interface. Even if the library does not have Python interface it 
is possible to use the tools as e. g. ctypes or Cython to call libraries functions directly from Python.

\paragraph{NumPy}

NumPy is open source Python library, which supports multidimensional arrays and many operations, which can be
performed with this arrays including linear algebra operations. Killer feature of NumPy is its speed,
which allows to write very effective code with just few lines. Many operations which NumPy do can be 
also rewritten by Python code using  for loops. However if NumPy function is called,
computation is perform by highly optimized C code, which makes big difference if arrays contain 
lot of elements. Sometimes it is difficult for beginner to figure out the way how 
to write problem using NumPy avoiding for loops. However as a user gets experienced 
it become easy to do it. 

NumPy is so widely used Python library that it has become one of fundamental libraries, thus many
other libraries and application has already rely on NumPy requiring its installation in order to work.

\paragraph{SimPy}

Simpy is another Python library which is focused on process-based discrete-event simulation.
Among many other features it supports symbolic expression. This symbolic expression can be useful
e. g. for computation of derivation of an function. The result of derivation is symbolic expression which can 
be than used for retrieving numeric values. 

SimPy is used for generation of partial derivatives of collinearity equation. Thanks to the SimPy 
it is easy to add new parameters into collinearity equations as distortion or leverage arm offset,
because it is just needed to extend collinearity equations and the source code of partial derivatives 
is generated automatically by Simpy. 

\subsubsection{GRASS Programming  environment}

Architecture of GRASS 7 is composed of core libraries and modules. There is a general GIS library, which 
deals with basic setup of GRASS environment, which includes management of
GRASS variables and environment variables. These variables are very important, because GRASS is not one monolithic software 
but it comprises of many small software packages called modules. When the module is run, essential information 
about GRASS session (current location, mapset etc.) is passed through these variables. 

Another main core libraries are dealing with different GIS data e. g. raster, vector or imagery library etc. 
All core libraries are written in C. In recent years support of Python programming language has been significantly 
improved. The main motivation was to make development of GRASS accessible for wider number 
of people, because Python is very simple and popular programming language. In most cases using 
C for development of module is overkill, which cost developer much more time 
with no significant gains compared to writing it Python. Nowadays the Python is supported very well in GRASS.
GRASS C libraries functions can be accessed from Python through ctypes interface, which is automatically generated during compilation
for all GRASS libraries.  
There also exists 
PyGrass API, which allows calling C functionality from Python taking full advantages of Python beauty.
The last option is calling modules from Python. Unlike libraries GRASS module layer is more abstract, 
therefore some specific functionality may be missing compared to the libraries. Calling of module
compared to calling of library function is slower.
On the other hand some functionality is implemented
only in module layer, and therefore this is only one possibility how to access it.
Big momentum has been given for integration of Python into the GRASS by development of wxPython graphical
user interface. 
Thanks to the Python support, GRASS has attracted new developers which has implemented many new modules.

\subsection{Modification of GRASS GIS orthorectification workflow}

The implementation should fulfill these goals:

\begin{itemize}
\item Backward compatibility, which preserves already implemented functionality, which
can be still useful in some cases. 
\item Take advantage of modularity of i.ortho.photo module.
If the workflow is comprised of modules/submodules with clearly defined interface, it is possible to implement
several different algorithms for every step of orthorectificaion. The main advantage of this 
approach is that workflow can be adapted to the specific needs, which allows to use 
most suitable algorithms to achieve maximum accuracy.
\item There should be another more abstract module for the less advanced users, which do not have sufficient knowledge to create individual 
worflow.
The module is easy to use wrapping the more specific modules.
\item Eventually graphical user interface should be developed. The interface should 
allow to identify tie points and ground control points, visualize the results and help with assessment of results accuracy.  
\end{itemize}

If all this goals would be fulfilled GRASS GIS would become unique flexible state of the art orthorectificaion 
software, which could be used by wide spectrum of users from amateurs to experts.
Due to time limitation this thesis deals only with implementation of functionality of some of the lowest level modules. 
However it is very important to have 
the ultimate goal in mind to properly design solution from basis in order to save a time in future,
which would be spend on refactoring in case of bad design.

In order to meet the implementation requirements the whole processing chain can be implemented in form
of these GRASS low level modules:

\begin{itemize}
\item Camera calibration modules - Retrieves interior orientation, which is supposed to be known by 
next steps of the workflow, therefore this step can be represented be the calibraiton module.
\item Point match modules - Identification of tie points in the acquired images.
\item Intial values modules - These modules computes initial value of exterior orientation and 
object coordinates of tie points, which are used in BBA. It uses information from the both
previously mentioned modules.
\item BBA modules - The module perform BBA based on data obtained by the other modules described in previous steps.
\end{itemize}


\subsubsection{Camera calibration module}

The calibration module i.ortho.cam.calibrate is simple module written in Python, which calls
function from OpenCV calib3d module. The module is based on calibration from 2D patern, which can 
be easily printed. The most important input parameter defines directory with images of 2D pattern, 
taken by the camera. In order to obtain good results of camera calibration, at least 10 convergent images 
from different positions all around the pattern should be taken \cite{camera_calibration2013opencv}.

Currently OpenCV supports these three patterns:
\begin{itemize}
\item Chessboard pattern 
\item Asymmetric pattern of circles
\item Symmetric pattern of circles
\end{itemize}

Generally the circle patterns achieves higher accuracy \cite{camera_calibration2013opencv}.

The identification of pattern points (centers of circles or corners of chessboard squares) 
are done by OpenCV functions, which are able to identify the points on photo of the pattern 
without any other additional information except number of circles or squares per row and column
of the pattern. Then the points coordinates in the photo are refined to achieve subpixel accuracy.

Except of photos no other information is needed to retrieve interior orientation of camera.
Object coordinates are defined
arbitrary by the module according to pattern type and number of rows and columns of pattern marks.

At first  initial exterior orientation of photos and interior orientation of camera is retrieved 
by closed form solution exploiting condition that all points are on the plane.
Interior orientation is calculated ignoring distortion.

In the next step initial values of distortion parameters are calculated by linear least squares method, 
where other orientation parameters except the distortion are kept fixed.

In the last step iterative BBA adjusting all interior orientation and exterior orientation parameters 
is performed.

The method is in-depth described 
in \cite{zhang2000flexible}.

The computed interior orientation is saved into GRASS GIS camera file, which is used by other GRASS GIS 
orthorectificaiton modules. The module modifies the camera file via i.ortho.camre module.
The GRASS GIS camera data model was extended to support radial and tangetial distortion coefficients, which 
have serious effect especially in cheap cameras, where optic system is not so precise.

\subsubsection{Bundle block adjustment data model}
 
Both the initial values and BBA modules basically handles  same data, thus 
the data model can be designed in similar way. There can be 
identified three essential classes, which are shared by both modules :
\begin{itemize}
\item Camera - the camera class has to store information about interior orientation 
parameters including distortion. Camera class should contain references of the photos
classes, which were acquired by the camera.
\item Photo - members of photo class are references to the points, which were identified 
	    in the photo and camera which take the photo,
\item Point - the point class denotes representation of same object point in different images (tie points).
 If the point is GCP it includes also information about it's object coordinates. 
\end{itemize}


This classes offer enough flexibility for adaptation to needs of both modules, which can be done be done by subclassing. 
Also it sufficiently expresses  structure of the data and relations of the data structure elements.

\subsubsection{Initial values module - i.ortho.initial}

The i.ortho.initial module retrieves initial values of photos exterior orientation and initial tie points object coordinates. If at least 3 gcps are 
present in an input, relative coordinate system is transformed into world coordinate system of the gcp by applying Helmert transformaiton. 

The module input is: 
\begin{itemize}
\item Interior orientations of cameras e. g. computed by i.cam.calibrate
\item Tie points photo coordinates 
\end{itemize}

The module builds internally graph like structure, where nodes represents photo and edges holds information 
about number of corresponding tie points, which are identified on two photos which edge links together.
Then the relative orientation of photo pairs is performed. The photos are oriented in order defined by going through the graph. 
Reference pair of photos which is taken for the first relative orientation, is comprised by the photos with the highest number of the corresponding tie points.
The next candidate is chosen from the subset of neighborhood photos, which have not been already relatively oriented. The chosen pair 
must also satisfy condition of the highest number of corresponding tie points from all candidate pairs. The pair is comprised of one photo, which is already 
part at least in one relative orientation, and the second photo, which is newly oriented. The relative orientation of the photo 
pair is performed by  the five point algorithm, which has been recently added into development version of OpenCV as function findEssentialMat \cite{calib_manual2013opencv},
which returns the essential matrix. Exterior orientation is retrieved \ref{ess_eo} from the essential matrix by Rodrigues \cite{calib_manual2013opencv} 
function. Because the exterior orientation is defined by relative coordinate system of the two cameras, it is needed to be transformed into 
the common relative coordinate system. The common system is defined by the relative orientation coordinate system of the reference pair, with highest 
number of tie points.
Every other relative orientation system is connected to the common system by process of merging of relative orientations see \ref{sec:ess_chain}.

The last optional step of the module is transformation of the relative coordinate system into the word coordinate system. Helmert 
transformation coefficients \ref{sec:helmert} are retrieved by code from v.rectify module \cite{v.rectify}, 
V.rectify module is used e. g. for transformation of Lidar point clouds.
The code, which computes the coefficients
was moved from the v.rectify module into the GRASS GIS imagery library in order to be accessible also by other modules as i.ortho.initial
module, which uses ctypes interface to call it.
Calculated Helmert transformation coefficients are applied on the exterior orientations of cameras and object coordinates of the tie points.

The output of the module are exterior orientations and object coordinates in common relative or world coordinate system. Together with interior
orientation parameters and image coordinates all needed information is obtained for the i.ortho.bba module which performs final BBA.  

\subsubsection{Bundle block adjustment module - i.bba}

The bundle block adjustment module performs non-linear least squares iteration, which 
uses collinearity equations as cost functions. 
The module input represents all parameters of collinearity equations:
\begin{itemize}
\item Tie points photo coordinates e. g. identified manually or automatically by points matching algorithms
\item Interior orientations of cameras e. g. computed by i.cam.calibrate
\item Exterior orientations of cameras e. g. initial values can computed by i.ortho.initial
\item Tie points object photo coordinates e. g. initial values can computed by i.ortho.initial
\end{itemize}

The module performs non-linear least square iteration until correction of the parameters from the adjustment are smaller than
defined threshold. The module is able to perform non-linear least squares as free network or using selected points 
as knowns (at least three) and thus solve it in classical way. 

The output of the module is the adjustment protocol. In the protocol, there
is described every iteration of the bba including correction for all adjusted parameters and standard deviations computed from 
covariance matrix of the adjustment. 

\section{Test case}

The processing chain was developed and tested on data, which were kindly provided by Martin Řehák. The available 
information about scene included:

\begin{itemize}
\item Tie points identified on the photos
\item Position of cameras (leverage arm offset + gps coordinates)
\item GCPs
\item Interior orientation of camera, which acquired the photos of the scene
\end{itemize}

Every parameters also included its standard deviation. 

Also there was included protocol of BBA from Bingo-F software \cite{bingo2013gip}, which reveals 
information about processing steps which the software uses to retrieve structure of the scene.
In general information from this protocol was very helpful for me, because 
 it showed me how the processing chain should look like.

Bingo-F is widely used, well tested software, which is used by many users on various scenes configurations.  
Compared to the developed processing chain in this thesis it is miles ahead, therefore the results of the Bingo-F adjustment were 
regarded as reference for obtained data from the developed processing chain.

Generally Bingo-F processing is very similar to the developed  processing chain, which is  
described in-depth in previous chapters. First step is the relative orientations of pairs, then the pairs are merged into common relative 
coordinates system. After that helmert transformation is performed into world coordinate system and as a last step BBA is performed.

Unlike the developed processing chain,  the Bingo-F software also take into account information about position of cameras. 
Because the developed processing chain should be general as much as possible it was decided that in first version  the processing chain would ignore this 
information because according to the set up goals the final solution should require only photo coordinates of tie points and interior 
orientation of cameras. Anyway in future it is possible to extend the processing chain to work also with such additional information to improve accuracy 
of the results. 

Also the Bingo-F is capable to identify and rule out blunders in every step of processing chain and thus it is more robust.
These blunders check are noticeable from adjustment protocols in these steps:
\begin{itemize}
\item Interior and exterior orientations of photos
\item Object coordinates of tie points and gcps
\end{itemize}



Bingo-F is highly optimized therefore it allows to solve scenes with thousands of tie points.

Output  protocols of Bingo-F adjustment contains: 
\begin{itemize}
\item Relative orientation - It seems that Bingo-F asses reprojection error, which is computed 
as difference of measured photo coordinates of a tie point and reprojected object coordinate of the tie point computed during relative orientaion.
\item Merging to common relative system -  There is performed so called "Pre-adjustment" of some points. It is probably some type of freee network 
BBA. There exists also blunder check in computation of scale. There is checked difference of object coordinates of points 
which were used for computation of scale. The difference is computed between tie points defined in both common relative system 
and relative orientation system of merged pair after being transformed into the common relative system.
If standard deviation of object points coordinates is higher than certain threshold, then 
scale is computed again skipping point with the highest difference.
The process is repeated  until  the threshold of standard deviation is satisfied.  
\item During BBA iteration there are probably used weight where information about standard deviation of measurement 
and standard deviation of parameters is utilized. Blunder detection mechanism rules out points, which 
has standard deviation higher than certain threshold. Setting standard deviation threshold  is common way how to identify 
blunders. It comes out of the normal distribution characteristics, because probability that measurement occurs with value 
 higher than 3 standard deviations is extremely small, therefore such a measurements are considered as blunders in practice.
\end{itemize}

The scene was adjusted as free network adjusting interior orientation parameters (focal length, principal point coordinates),
exterior orientations of photos and object coordinates of all points. It was also adjusted lever-arm offset which describes 
shift of GPS unit from projection center.  Lever-arm offset allows Bingo-F to use information of photo positions from GPS unit,
which is shifted by this offset.  

Final reconstructed  scene from Bingo-F results is shown in this plot:

\includegraphics[scale=0.5]{figures/bingo_result.png}

Unlike many other real scenes which can be comprised by hundreds of photos and thousands of tie points,   
there are just a few tie points (64) and photos (45) taken by same camera. Also the photos capture same points thus there 
is lot of redundancy. Solving the scene with such redundant scene should improve robustness of the BBA iteration. 

In the rest of the chapter there will be presented results of subsequent steps, which are performed in processing chain and compared 
to the Bingo-F results.

The first step is relative orientation of pairs. The relative orientations of the pairs were done by OpenCV 5 point algorithm.
The algorithm was successful on most of the pairs, however there are some pairs were algorithm does not work well.

Example of the algorithm success can be showed in this plot:

\includegraphics[scale=0.4]{figures/rel_or_576_598.png}

The right plot shows the result of five point algorithm and the left plot shows result of Bingo-F. Note that both 
plots are in different coordinate system, however it is possible to asses it visually. It is clearly visible
that both scenes are very similar.

On the other hand, there exists  few cases, with unsuccessful relative orientation.  
One of such a cases is relative orientation of photos pair 553 and 591:  

\includegraphics[scale=0.4]{figures/rel_or_553_591.png}

It is clearly visible that result of relative orientation is completely different from the Bingo-F result.
Most important factor, which affects success of the five point algorithm is threshold parameter of RANSAC defining 
if point is considered inlier or outlier.
The above mentioned relative orientations were computed with RANSAC threshold parameters set to 0.001 mm.
If the threshold parameter is changed in 0.1 mm, the result of relative orientation of pair 553 and 591 improves significantly.
On the other hand another relative orientations go wrong. Therefore with statically set threshold parameter, there are always 
few evidently wrong relative orientations in this scene. Thanks to the configuration of the scene, where lot of photos are 
covering same points, it is  not serious problem, because the point is determined in most relative orientations  with 
much better accuracy, thus it is possible to exclude this outliers. This could be serious problem e. g. if the photos 
of scene would cover long chain where redundancy is much more lower.   
In this case one wrong relative orientation could thread whole result of bba. If error would appear in the middle 
of the chain it would  be propagated up in one half of the chain which transformation depends on the wrongly
determined pair, which is apparent from equations in \label{eq:comm_rel}. 
In the test case there are not so many critical pairs, which has to be determined correctly to avoid spoiling the transformation 
of the others dependent pairs. Fortunately the reference pair, which defines common relative system, is very stable and also 
the other  core pairs on which depends lot of other pairs are stable. 
Probably there is some correlation between number 
and success rate of RANSAC loop to select correct solution. 
Therefore another way how to improve stability of the solution 
could be identification of more tie points, which could be done automatically by pattern matching algorithms. 
In the test case there 
are 27 tie points in the reference pair. Which gradually decreases to 6 point in the pair, which is merged as last.

During the merging process all object coordinates, which are computed from different relative orientations and represent same point, are 
transformed into  the common relative system individually. Then the final coordinates of object points are determined as median
of corresponding transformed object coordinates from different relative orientations. Median was chosen because it is more robust to the effect of 
big outliers caused by few wrong relative orientations in the set. If there is more than 50 percent of coordinates 
which are close to the correct value and the rest are big outliers (luckily the test case is the case), median gives reasonable results over average,
which is much more sensitive to the effect of the big outliers. 

This plot shows situation before computing median of all corresponding object coordinates:

\includegraphics[scale=0.55]{figures/before_median.png}

Plotted points with same color represents same the corresponding coordinates computed by different relative orientation. It can be noticed that 
the points of same color tend to make clusters, however there are some big outliers, which effect is eliminated by applying
median on the points.

The next step is helmert transformation of the common relative system into the world coordinate system using available GCPs.
The transformed scene can be finally compared to the results of Bingo-F adjustment.
Average difference of exterior orientation 
of camera is approximately around 0.5 meters with highest outlier in meters. The object points coordinates 
are little bit more accurate with average approximately 20 cm and highest outliers in 3-1 meters. 
Angles differences are shown in this plot, where there are plotted axes of two camera coordinates systems, which represents 
Bingo-F adjustment angles and the  initial angles. 

\includegraphics[scale=0.4]{figures/angles_comaprison_relative_orinetaion.png}

It is clear that there are a few orientations, which differs in tenths of degrees form the result. 

As a last step BBA was applied on these initial values with taking GCP as knowns.
Despite the fact that there were few huge outliers in both parameters of cameras and points,
BBA after five iteration subsequently refines all scene parameter and reduces all cordinates 
differences into the centimeters and angles into the tenths of degrees. 


This plot again shows differences of cameras coordinates system, which are no longer visible from this point of view
unlike to the previous plot:

\includegraphics[scale=0.4]{figures/result.png}


\section{Future development}


Basically the work which has been done in this thesis, can be considered as prove of concept,
which shows path of future implementation of BBA in GRASS GIS. The developed 
processing chain in current state cannot be considered as working solution.

Currently  the processing chain was just tested 
 on one scene, where it gives reasonable results. However 
author is sure that in other more complicated scenes it would not work successfully because 
the solution is not robust at all.

In order to improve robustness of the processing chain this suggestions should be considered:
\begin{itemize}
\item Relative orientations - there is no check, whether relative orientation accuracy is successful or not. There 
should be implemented some mechanism, which asses quality of relative orientation (probably according to reprojection error of object points).
Also it should be examined work of the five point algorithm with RANSAC loop on scenes with higher number of the tie points, where 
the outliers elimination effect of RANSAC loop should be much stronger.
\item  Merging of relative orientations - currently if there is one wrong relative orientation, with lot of depended relative orientations,
the error gets propagated up during transformation of depended orientations to the common relative coordinate system. 
Effect of such a inaccurate pair could be 
reduced by performing  temporary BBA, which can fix these some of big inaccuracies. Also the order of relative orientations in merging 
into  should not 
be chosen only by number of tie points but also taking into account quality of relative orientation.

Calculation of scale lacks checks 

\item During final BBA, the weights are ignored using identity matrix. Good choice of weights during every least squares iteration 
could be big leap towards robustness of the BBA. If the weights are selected properly, it is possible to eliminate of outliers 
effect.
Another way how to improve final accuracy of BBA is skipping points, which 
standard deviation from adjustment is higher than some threshold after. The standard deviation criterion can be assessed 
after every iteration.
\end{itemize}


Another big room for improvement is computational speed of the processing chain.
The most critical part which costs most of computational time is BBA.
One least squares  iteration of BBA takes approximately 5 seconds in the test case, which can be 
considered as very small scene compared to commonly the solved
scenes with hundreds of photos and thousands of points. The bottleneck of BBA iteration is computation of inverse/pseudo inverse, where big matrices 
must be solved. The size of the matrix is directly connected to the number of adjusted parameters. There exists  lot of methods which reduce computational 
time of matrix inversion. The inversed matrix of BBA is sparse (most of elements are zeros).

There exists many optimization methods exploiting this property. 
It is also possible to speed up some parts of the code reimplementing them in C instead of currently used Python. However this should be done 
as last step when the part, which is going to be reimplemented in C is tested and works well. Otherwise it would be waste of time and energy, if reimplemented 
code would not be eventually used. 

Another option should which should  be considered is employing of some free software BBA libraries.
Some of the free BBA libraries are:
\begin{itemize}
\item sba
\item Simple Sparse Bundle Adjustment 
\end{itemize}

Both libraries are highly optimized and they are based on Levenberg–Marquardt algorithm.
On the other hand it would require to add new dependency into GRASS, if such a library would be used. 
There is trend to avoid adding new dependencies into GRASS because it already relies on many software libraries,
which makes installation of GRASS sometimes difficult because of countless combinations of library versions.



There is still needed to define interface between the developed modules and also their inteface. 
This thesis outlines just rough architecture of final solution, because at this early stage of development 
there is no point in dealing with the interface, because it can change lot of time during further development.


\section{Conclusion}

\section{References}
\bibliography{DP}
\bibliographystyle{plain}
\bibentry{Hartley2004}

\bibentry{wiki:SIFT}

\bibentry{wiki:SURF}

\bibentry{leutenegger2011brisk}

\bibentry{barazzetti2010extraction}

\bibentry{wiki:Eight-point_algorithm}

\bibentry{stewenius2006recent}

\bibentry{bruckner2008experimental}

\bibentry{nister2004efficient}
TODO
\bibentry{pietzsch2001robot}
TODO
\bibentry{pietzsch2004application}

\bibentry{rocchini2012robust}

\bibentry{i.ortho.photo}

\bibentry{neteler2008open}

\bibentry{camera_calibration2013opencv}

\bibentry{zhang2000flexible}

\bibentry{calib_manual2013opencv}

\bibentry{v.rectify}

\bibentry{brown1966distortion}

\bibentry{bingo2013gip}

\bibentry{labe2004geometric}

\bibentry{mellish2005pinhole}

\bibentry{boe2013octocopter}

\bibentry{kbosak2010bezmiechowa}

\bibentry{baumker2001new}

\bibentry{sjoberg2013closed}

\section{Appendix}
\subsection{Adjustment protocol}
\label{sec:adj_protocol}


\end{document}}
