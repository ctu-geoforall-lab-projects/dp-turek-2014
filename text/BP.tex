\documentclass[a4paper,12pt]{report}
\usepackage{a4wide}
\usepackage{lmodern}
\usepackage[T1]{fontenc}
\usepackage{textcomp} 
\usepackage[utf8]{inputenc}
\usepackage{xcolor}
\usepackage[czech,english]{babel}
\usepackage[pdftex, final]{graphicx}
% \usepackage[pdftex, final, colorlinks=true]{hyperref}
\usepackage{verbatim}
\usepackage{alltt}
\usepackage{paralist}
\usepackage{mdwlist}
\usepackage{subfig}
\usepackage[final]{pdfpages}
\usepackage{amsmath}
%\usepackage[hyphens]{url}
%\PassOptionsToPackage{hyphens}{url}

\usepackage{bibentry}
\makeatletter\let\saved@bibitem\@bibitem\makeatother
\usepackage[final,pdftex,colorlinks=false,breaklinks=true]{hyperref}
\makeatletter\let\@bibitem\saved@bibitem\makeatother
\usepackage[hyphenbreaks]{breakurl}

%%%%%%%%%%%%%%%%%%%%%%%%%
% pro podmineny preklad
% false je defaultně


% \newif\ifbc % Pouze do bakalářské práce
%  \bctrue

%%%%%%%%%% fancy %%%%%%%%%%%
\usepackage{fancyhdr}

\fancyhead[L]{ČVUT v Praze}

\setlength{\headheight}{16pt}

% \usepackage{stdpage}


%%%%%%%%%%%% rozmery %%%%%%%%%%%%%%%%%%
\usepackage[%
%top=40mm,
%bottom=35mm,
%left=40mm,
%right=30mm
top=40mm,
bottom=35mm,
left=35mm,
right=25mm
]{geometry}


\renewcommand\baselinestretch{1.3}
\parskip=0.8ex plus 0.4ex minus 0.1 ex

\newcommand{\klicslova}[2]{\noindent\textbf{#1: }#2}
\newcommand{\modul}[1]{\emph{#1}}
\author{Štěpán Turek}
% \pagecolor{darkGrey}
\newcommand{\necislovana}[1]{%
\phantomsection
\addcontentsline{toc}{section}{#1}
\section*{#1}

\markboth{\uppercase{#1}}{}
}

\newcommand{\ematr}[1]{
{\bf #1}
}

\newcommand{\evect}[1]{
{\bf #1}
}

\newcommand{\ehvect}[1]{
{\bf \widetilde{#1}}
}

\newcommand{\escal}[1]{
{\it #1}
}

\newcommand{\eucl}[1]{
{\bf R\textsuperscript{#1}}
}
\newcommand{\proj}[1]{
{\bf P\textsuperscript{#1}}
}

\newcommand{\efunc}[1]{
{\it #1}
}


%%%%%%%%%%%%%%%%%%%%%%%%%%%%%%
\begin{document}
\pagestyle{empty}

\begin{center}
%napisy

%%% ML: diplomka je v anglictine, potom by i uvodni strana mela byt v
%%% ang, viz
%%% http://geo.fsv.cvut.cz/proj/dp/2013/anna-kratochvilova-dp-2013.pdf
%%% ST: opraveno

%%% ML: u jmena si dopln titul (Bc.)
%%% ST:doplnen

\newcommand{\napisCVUT}{Czech Technical University in Prague}
\newcommand{\napisFS}{Faculty of Civil Engineering}
\newcommand{\napisObor}{Branch Geoinformatics}
\newcommand{\napisKatedra}{Department of Geomatics}
\newcommand{\napisVedouci}{Ing. Martin Landa, Ph.D.}
\newcommand{\napisAutor}{Bc. Štěpán Turek}
\newcommand{\napisDatum}{Mosty u Jablunkova 2013}
\newcommand{\napisNazevI}{Implementation of bundle block}
\newcommand{\napisNazevII}{ adjustment method for determination of}
\newcommand{\napisNazevIII}{exterior orientation into GRASS GIS}
\newcommand{\napisNazevAjI}{Implementace metody svazkového vyrovnání bloku}
\newcommand{\napisNazevAjII}{pro určení prvků vnější orientace do programu}
\newcommand{\napisNazevAjIII}{GRASS GIS}
\newcommand{\napisBakalarka}{Master’s thesis}
\newcommand{\napisPraha}{Mosty u Jablunkova 2013}


%
% prikazy
%\newcommand{\velka}[1]{\uppercase{#1}}
\newcommand{\velka}[1]{\textsc{#1}}
%
% 
\newif\ifpatitul
\patitultrue

\ifpatitul
{\Large\velka{\napisCVUT}}\\
\velka{\Large\napisFS}\\
\vfill
{\LARGE\velka{\napisBakalarka}}
\vfill
{\large\napisPraha\hfill\napisAutor}
\newpage
\fi%patitul



{\Large\velka{\napisCVUT}}\\
{\Large\velka{\napisFS}}\\
{\Large\velka{\napisObor}}
\vfill
\includegraphics[width=3cm]{logo_cvut_cb} %~
\vfill
{\Large\velka{\napisBakalarka}}\\
{\Large\velka{\napisNazevI\\
\napisNazevII\\
\napisNazevIII}}\\
{\large\velka{\napisNazevAjI\\
\napisNazevAjII\\
\napisNazevAjIII}}
\vfill
{\large%
Supervisor: \napisVedouci\\
\napisKatedra\\
\bigskip
\napisDatum\hfill\napisAutor}
\end{center}

\newpage
\definecolor{navodotisk}{RGB}{10,10,10}
\newcommand{\vlozZadani}{%
\Huge\textcolor{navodotisk}{\textsf{\textbf{ZDE VLOŽIT ORIGINÁLNÍ ZADÁNÍ}}}%
}
%%%\includepdf[picturecommand={\put(100,200){\vlozZadani}}]{zadani}
 % resi si zalomeni sam

\begin{abstract}

\bigskip

\klicslova{Klíčová slova}{GIS, GRASS}

\end{abstract}

\selectlanguage{english}
\begin{abstract}

\bigskip

\klicslova{Keywords}{GIS, GRASS}

\end{abstract}
\selectlanguage{english}


\newpage
\newcommand{\odsaditodzhora}{\hskip1pt\vfill}

\odsaditodzhora
\noindent Declaration of authorship

I declare that the work presented here is, to the best of my knowledge and
belief, original and the result of my own investigations, except as acknowledged.
Formulations and ideas taken from other sources are cited as such.

December 21, 2013

\begin{flushleft}
\begin{tabular}{cp{0.3\textwidth}c}
In Mosty u Jablunkova, 
& 
&
..................................
\\
&&
(author’s signature)
\end{tabular}

\end{flushleft}
\newpage

%%% -> ML: chybi ;-)

\odsaditodzhora
\noindent Poděkování


\newpage

\newpage
\tableofcontents


\newpage
\pagestyle{fancy}

\necislovana{Introdduction}

In recent decades tremndous development of information technology has fundamentally changed many disciplines.

Photogrammetry has sucessfuly taken advantages of this development, and nowadays 
it is using complently different methods and equipment, which allows to obtain and process data 
in huge scales and achieve accuracy

The core of photogrammetry is moving  from  hardware towards software. In past it was necessary to have special 
photogrammetric equipment to perform photogrammetric measurement.  Thanks to current state of the art algorihms 
nowadays it is possible to use higher level digital cameras two perform photogrammetric measurements achieving
high accuracy. 
Thanks to this shift photogrammetry is not longer being domain of few proffesionals however it is 
becoming used be wide audience. 

We are just at the beggining of the golden age of photogrammetry, 
which with spread of UAV and development of augument
reality is going to find a way to our cell phones and other devices we use on daily bases. 

Heavy usage of information technologies also caused that border between disciplines is blurring. 
Photogrammetry was affected by computer vision, which evolved independetly until 2000's. After that it is possible 
to see growing trend combine knowledge from both fields.

With growing importance of software there comes a question of licenses. There are two 
main branches proprietary software and free software. Most of proprietary software are not available in 
form of source code. The license of proprietary sofware allows use it under certain terms and usually user 
has to buy it.


\newpage

\chapter{Theoretical part}


\section{GRASS GIS}

\section{UAV}

Unmanned aerial vehicle is aircraft without human pilot. UAV can be controlled by remote control.
Somo of the UAV are capable of autonomous flight, which is used when connection with remote control is 
lost or 

TODO notation

\section{Least squares}
\label{sec:least}


The main aim of the least square method is to find optimal solution from rerdurant measuremnts.


The least square method has two main requirements:
\begin{itemize}
\item Formulation of cost functions. Variables of cost function are parametrs, which get adjusted 
      during least squre method. Paramers are the measured one quantities.
      The core idea of least squares is finding such a values of parameters to minimize sum 
      of square diferences of optimal function results using adjusted parameters and measured one.
      
      Mathematially it can be written as getting global minumum of cost f 
      
\item Having more measuremnts than number of parametrs in cost function.

\end{itemize}


Principle of the method  will be derived using calculus.
As input to the least square there are these argmunets:
\begin{itemize}
\item \evect{L} - vector of measurements
\item \evect{X} - vector of parameters, which values are computed to during least square, intial values in
		  case of linear cost function can be random
\item   - linear cost functions, which describes measurements depending on parameters in \evect{X},
          which are arguments of the function
% 				  \efunc{f_{i}(\evect{X})}TODO
\item \ematr{A} - it is matrix of the const function  coefitients,  where row represents one measurent 
		  and column represents one parameter of cost function
\end{itemize}
		  
Differneces of measured value and values computed from parameters using cost function is:
\begin{equation}
\evect{v} = \evect{L} - \ematr{A}\evect{X}
\label{eq:least_v}
\end{equation} 
As it was already mentioned the least square method minimizes sum of square differences:

\begin{equation}
\evect{v}_{T} \evect{v} = min
\end{equation} 

\begin{equation}
\begin{split}
\evect{v}_{T} \evect{v} &= (\evect{L} - \ematr{A}\ematr{X})_{T} (\evect{L} - \ematr{A}\ematr{X}) \\
&= \evect{L}_{T} \evect{L} - 2 \evect{L}_{T} \ematr{A} \evect{X} + \evect{X}_{T} \ematr{A}_{T} \ematr{A} \evect{X}
\label{eq:least_be_part}
\end{split}
\end{equation} 

The minimum of the function can be get using partial derivative of parameters in cost function.
In this case derivative are simple becouse it is supposed that our cost functions are linear,
therfore there is derivated second degree polynomial.
 
\begin{equation}
\frac{\partial diff}{\partial \evect{X}} = -2\ematr{A}_{T} \evect{L} + -2\ematr{A}_{T}\ematr{A} \evect{X} 
\label{eq:least_part}
\end{equation} 

Giving this equatin equal to zero, the minimum is found:
\begin{equation}
-2\ematr{A}_{T} \evect{L} + -2\ematr{A}_{T}\ematr{A} \evect{X} = 0 
\end{equation} 

Last step is expression of parameters vector from the previus eqaution:

\begin{equation}
\evect{X} = (\ematr{A}_{T} \ematr{A})_{-1} \ematr{A}_{T} \ematr{L}
\end{equation}

This equation is the hearth of the least square method.

This equations suppose that all mesurents has same weigh however it is possble to extend the model to support weights. 
For example when in least square adjustment there can be combined the meausuremtns with different accuracies,
it is possible to characterize these accuracies creating weghts matrix.


Weight matrix \ematr{P}  is matix with weights of the measuremtns on the diagonal with off-diagonal zero elements.
Row of weight determines to wich measurement in desing matrix \ematr{A} it belogns.

The equatin which is least squares minimizes changes adding weight matrix into this form:
\begin{equation}
\evect{v}_{T}  \ematr{P} \evect{v} = min
\end{equation}

therefore:
\begin{equation}
\evect{X} = (\ematr{A}_{T} \ematr{P} \ematr{A})_{-1} \ematr{A}_{T} \ematr{P} \ematr{L}
\label{eq:least_sq}
\end{equation}


\subsection{Non-linear least squares}
\label{sec:non_least}
So far it was supposed to be cost function linear. However many problems, which  are refined using least square problem 
are not linear.

Very important consequence for least square method of non-linear cost functions is that result of partial derivatives 
\eqref{eq:least_part} does not have exactly one solution, which gives one global minumum, which can be 
elegantly solved by means of linear algebra \eqref{eq:least_sq}. 

It is clearly visible from equation \label{eq:least_be_part}, which is partially derived. This equation 
in case of two parameter linear \eucl{2} function represents a parabole. 
A parabole has only one mininum which is global. Beside this minimum there are no other points 
(local minimas, stationary points) on this function,

Same applies \eucl{n} e g. In \eucl{3} 
the function is represented by paraboloid which has also only one local minimum.
where derivative is zero. Therefore it has only one exact solution. 

In case of adjusting non-linear fucntions, the function \label{eq:least_be_part} is not parabole, however 
it is functuo

This problem of non-linearity can be solved by linearization process.
The main idea of linearization is to aproximate the cost function with first degree taylor polynom,
which is linear and therfore it is possible to use linear least square equation \eqref{eq:least_sq} 
to solve non-linear problem. 

The accuracy of parameters also determines how many interations (speed of convergence) are needed to achive optimal solution (global minimum).


Apprximation of cost function with first degre taylor polynom can be written as:
\begin{equation}
\efunc{T^{f_{i}, X_{0}}_{1}} = f_{i}(X_{0}) + \frac{\partial \efunc{f_{i}}}{\partial {X_{0}}} f_{i}'(X_{0}) (X -  X_{0}) 
... \frac{\partial \efunc{f_{i}}}{\partial {X_{0}}} f_{i}'(X_{n}) (X -  X_{0}) 
\end{equation}

where:
\begin{itemize}
\item $X_{0}$ is point where the taylor polynom touches aproximated function. The first degree taylor polynom in \eucl{2} 
      is line which is tangent of funtion at point $X_{0}$. 
\item X is point for getting the function value
\end{itemize}


Tht taylor polynom can be rewritten as:

\begin{equation}
\efunc{T^{f_{i}, X_{0}}_{1}} = f_{i}(X_{0}) - a_{0} x_{0} ... a_{n} x_{k} 
\end{equation} 

\begin{equation}
\efunc{T^{f_{i}, X_{0}}_{1}} = f_{i}(X_{0}) - a_{0} x_{0} ... a_{n} x_{k}
\end{equation} 

Which can be written as:
\begin{equation}
\efunc{T^{f_{i}, X_{0}}_{1}} = \evect{v} = \evect{L_{X_{0}}} - \ematr{A}\evect{dx}
\end{equation} 

And analogous equation to \label{eq:least_v} can be derived:
\begin{equation}
\evect{v} = \evect{L} - \efunc{T^{f_{i}, X_{0}}_{1}}
\end{equation} 

\begin{equation}
\evect{v} =  \evect{L_{X_{0}}} - \evect{L} - \ematr{A}\evect{dx}
\end{equation} 

and giving:
\begin{equation}
\evect{v} = \evect{l} - \ematr{A}\evect{dx}
\end{equation} 



and therfore it is possible to use linear least sqaures \eqref{eq:least_sq} for adjustment 
with non-linear cost functions.

However the equation is slightly different, instead of vecotr \evect{X} there as used 
\evect{dx} and vector  \evect{L} is analogous to vector \evect{l}. In this two minor 
substitutions there is hidden major limitation of the non-linear least squares method, 
where prize for approximation is paid. 

As it was already mentioned the taylor polynom requires some point $\evect{X_{0}}$.
Usually this approximation is good in close surroundings of point $\evect{X_{0}}$.

Unlike linear least square method in order to be approximation good it is needed to 
have some initial values of the paramers. If the initial values are not accurate 
enough, it is possible to iteratively run linear least squares to get values 
of paramers, which are in global mininum of the cost function. 

Unfortunatelly the non-linear least square method has another one pitfall, which 
does not guarentee that it will iterate toward solution,
which satisfies minimum squares difference condition. It can iterate towards
local minumum, stationary point or 

This is caused by run-off of non-linear function, which has more points with zero derivative,
and therfore result depends where the non-linear least square iteration starts. 

\section{Short intruduction into photogrammetry}

Both photogrammetry and computer vision deals with set of images, which covers some scene 
and using varius technics from these fields we are able to determine configuration of the scene. 
Using this information, it is possible to gain various information about the scene. 

Using photogrammetry approach, it is possible to create orthophoto map, which is transformation of images from 
perspective projection of camera  into orthigraphich projection or it is possible to create digital terrain model.

Computer vision field uses it for determination of robots position. Very popular is also structure from motion method,
which is able to reconstruct 3D structure from set of images. In essence structure frm motion method is equivalent to the creation 
of digital terrian model.

The scene can be recoverd if exterior and interior orientation of photos is known.
Exterior and interior orientation together describes relation between object point and
it's corresponding image point.

\subsection{Interior orientation}

Interior orienation describes relation between measure coordinates system, which defines loation of measured 2D points in the image 
and correspondent 3D point in photo coordintate system.

Photo coordinate system is right-handed cartesian system with origin in projective point, z axis heading to the image plane. In aerial photogrammetry
x axis according to ISPRS is heading in direction of flight. 

Photogrammetr equations are based on pinhole camera model, which describes projection of 3D point onto image plane. 
However every camera deviate from this model, therefore it is necessary to apply corrections for images taken by the camera,
which are distorted  as a consequence of the model deviation effect.
After applyication of the corrections it is possible use pinhole camera model.

Interior orientations parameters describing pinhole camera model are:
\begin{itemize}
  \item Focal length - it is distance of projective center from principal point
  \item Principal point coordinates - In ideal case location of principal point would exactly in the middle 
	of the image.  
	However in most cases there is some shift of principal point from ideal case.
\end{itemize}


Interior orientations parameters describing distortion are:
\begin{itemize}
  \item Radial distortion - it is caused by diferences in lateral magnification of lens. Radial distance is measured from principal point.
  \item Tangential distortion - it is produced by inaccurate centration of lenses and it is peprendicular to radial distortion.
\end{itemize}


\subsection{Exterior orientation}

After transformation of 2D image points into photo coordinate system it is usually necessary to transform it into photo object coordinate space.


Object space coordinate system must be cartesian. Ig we have points in non-cartesion system, it should be always transformed into cartesian 
system before using in bellow mentioned equations.


Exterior orienation is defined by position of camera center in object space and orientation of photo coordinate syste.
Photgrammetry uses representation of orientation by euler angles and rotation matrix.

The euler angles are group of three angles, which describes susequent rotaitons
 around axes of 3D coordinate system. 
 Order of euler angles is important, because in case of breaching it, the result of rotation is different. 
 In this work there is used BLUH notation when at first rotation around y axis is 
 perform, than system is rotated around y axis and the last rotation is around z axis.
 
 TODO rotation matrices equation:
 
\subsection{Basic formula of photogrammetry}

Combining of exterior orientation and interior orientaiton it is possbile to define 
direct relation between corresponding 3D point in object space and 2D point
in image space.

\begin{equation}
\label{eq:col_eqs}
\begin{split}
&\escal{x}^{im} = -\escal{f}\frac{\ematr{R}_{11}(\evect{X}^{obj} - \evect{X}^{obj}_{pp}) + 
                                  \ematr{R}_{12}(\evect{Y}^{obj} - \evect{Y}^{obj}_{pp}) + 
                                  \ematr{R}_{13}(\evect{Z}^{obj} - \evect{Z}^{obj}_{pp})                                  
                                  }{
				  \ematr{R}_{31}(\evect{X}^{obj} - \evect{X}^{obj}_{pp}) + 
                                  \ematr{R}_{32}(\evect{Y}^{obj} - \evect{Y}^{obj}_{pp}) + 
                                  \ematr{R}_{33}(\evect{Z}^{obj} - \evect{Z}^{obj}_{pp})     
                                  } \\
&\escal{y}^{im} = -\escal{f}\frac{\ematr{R}_{21}(\evect{X}^{obj} - \evect{X}^{obj}_{pp}) + 
                                  \ematr{R}_{22}(\evect{Y}^{obj} - \evect{Y}^{obj}_{pp}) + 
                                  \ematr{R}_{23}(\evect{Z}^{obj} - \evect{Z}^{obj}_{pp})                                  
                                  }{
				  \ematr{R}_{31}(\evect{X}^{obj} - \evect{X}^{obj}_{pp}) + 
                                  \ematr{R}_{32}(\evect{Y}^{obj} - \evect{Y}^{obj}_{pp}) + 
                                  \ematr{R}_{33}(\evect{Z}^{obj} - \evect{Z}^{obj}_{pp})     
                                  }
\end{split}
\end{equation}

This two equatins are called collinerarity equations and they describe direct relation between image point and corresponding photo point.

\section{Leverage arm offset}


\section{Short introduction into computer vision}
\subsection{Homogenous coordinates}


In computer vision there are mostly used homogenous coordinates, which allows to express some relations
mathematically in more elegant way than using cartesian coordinates. 

3D point in homogenous coordinates is defined as:

\begin{equation}
\ehvect{x} = (\escal{x}, \escal{y}, \escal{z}, \escal{w})
\end{equation}

\escal{w} componnet is called scale.

The tansformation of homogenous point into cartesian coordinates is done with division 
coordinates by scale:
\begin{equation}
\evect{x} = (\escal{x} / \escal{w}, \escal{y} / \escal{w}, \escal{z} / \escal{w})
\end{equation}

Using homogenous coordiantes it is possible to write all cartesian points plus points in infinity.
Infinite point in homogenous coordinates is written as: 

\begin{equation}
\ehvect{x} = (\escal{x}, \escal{y}, \escal{z}, \escal{0})
\end{equation}

Which is inpossible to express using cartesion coordiantes with finite numbers:

\begin{equation}
\evect{x} = (\escal{x} / \escal{0}, \escal{y} / \escal{0}, \escal{z} / \escal{0})
\end{equation}

TODO graphic interpretation of hom coords 

Thanks to homogenous coordinates for lines in \eucl{3} and point in \eucl{2} it is possible to use simple vector linear algebra operations 
do get useful information.


Plane in \eucl{3} can be written as:

\begin{equation}
\escal{a}\escal{x} + \escal{b}\escal{y} + \escal{c}\escal{z} + \escal{d} = 0
\end{equation}

Using homogenous coordinates it is possible to express it is as vector:

\begin{equation}
\ehvect{\pi} =  (\escal{a}, \escal{b}, \escal{c}, \escal{d})
\end{equation}


Line in \eucl{2} can be defined by equation:
\begin{equation}
\escal{a}\escal{x} + \escal{b}\escal{y} + \escal{c} = 0
\end{equation}

Using homogenous coordinates it is possible to express it is as vector:

\begin{equation}
\ehvect{l} =  (\escal{a}, \escal{b}, \escal{c})
\end{equation}

Using homogenous coordinates it is easy to find out whether point lies on line in \eucl{2}:

\begin{equation}
\ehvect{x}^{T} \ehvect{l} = 0
\end{equation}


Definition of line in \eucl{3} is not so nice as in \eucl{2}.



In homogenous it can be expressed with given points $\ehvect{p}$ and $\ehvect{q}$ as:
\begin{equation}
 \ehvect{l} = \escal{\mu}\ehvect{q} + \escal{\lambda}\ehvect{p}
\end{equation}

If the point is given as direciton of the line $\ehvect{d} = (x, y, z, 0)$ (it is infinite point), the equation get simplified:
\begin{equation}
\ehvect{l} = \ehvect{d} + \escal{\lambda}\ehvect{p} \label{eq:hline}
\end{equation}
TODO where not zeros

Simillar equation applies for point on plane in \eucl{3}:

\begin{equation}
\ehvect{\pi}^{T} \ehvect{l} = 0
\end{equation}

Computation of intersection of lines ($\ehvect{l}, \ehvect{l}'$) can be done also in very simple way by cross product:

\begin{equation}
\ehvect{x} = \ehvect{l} \times \ehvect{l}' \label{eq:def_hline_pts}
\end{equation}


Note that with homogenous coordiantes it is possible to express itersection of paraler lines, which lies in infinity, 
therfore it's scale is equal to 0.

TODO dualism

Simillarly line can be derived as cross product of two points:

\begin{equation}
\ehvect{l} = \ehvect{x} \times \ehvect{x}'
\end{equation}


\subsection{Basic formula of computer vision}

Equivalent to collinearity equations computer vision uses this equation to describe relationship between
object space and image space:

\begin{equation}
\begin{pmatrix}
   \escal{x} \\
   \escal{y} \\
   \escal{1} \\
\end{pmatrix}
=
\begin{pmatrix}
   & \escal{f_{x}} & 0     & \escal{c_{x}}\\
   & 0     & \escal{f_{x}} & \escal{c_{x}}\\
   & 0     & 0     & 1\\
\end{pmatrix}
\begin{pmatrix}
   &\ematr{R} & \evect{t}\\
\end{pmatrix}
\begin{pmatrix}
   \escal{X} \\
   \escal{Y} \\
   \escal{Z} \\
   \escal{1} \\
\end{pmatrix}
\end{equation}

Which can be abbreviated as:

\begin{equation}
\escal{w} \ehvect{x} = \ematr{K} \ematr{[}\ematr{R}|\evect{t}\ematr{]} \ehvect{X}
\label{eq:p_exp}
\end{equation}

Whole projection can be merged into single matrix \ematr{P}, which is called camera marix:

\begin{equation}
\escal{w} \ehvect{x} = \ematr{P} \ehvect{X}
\label{eq:p_abbr}
\end{equation}

TODO picture of system

where:

\begin{itemize}
\item \ematr{K} is calibration matrix which contains interior orientaion parameters.  

\item \ematr{[\ematr{R}|\evect{t}]} matrix contains rotation and translation wich transforms point from object cordinate system 
	      into camera cordinate system, which is left handed, with z axis perpedicular to image plane, with
	      origin in principal point heading into the scene direction. Difference between camera coordiante 
	      system in computer vision and photo coordiante system in computer vision is in deriction 
	      if z axis and therfore in handness of system. Camera cooridnate system in computer vision 
	      is left handed and photo coordiante system is right handed. TODO 
	      Vector $\evect{t}$ can  be transformed in object space coordiantes system:
	      \begin{equation}
	      \ehvect{C} = R_{T}\evect{t}
	      \end{equation}
	      giving coordinates of camera projective center. The parameters \ematr{R} and $\ehvect{C}$
	      are exterior orietation of camera
	      
\item $\ehvect{X}$ euclidean object space coordinates extended into homogenous form

\item $\ehvect{x}$ image space homogenous coordinates
\end{itemize}

Camera matrix \ematr{P} defines reprojection up to scale $\escal{\lambda}$, therefore every camera matrix multiplied by scale gives same image points:

\begin{equation}
\ehvect{x}=\lambda\ehvect{P}\ehvect{X}
 \end{equation}

It can be seen as representation of ray going from the projective center. The object point can be located everywhere on this ray, getting always 
same image point coordinates.

\subsection{Two view geometry}

In this chapter there will be introduced theory which describes mutual relation of two images. 
The core of the theory lies in epipolar geometry.

TODO picture of epipolar geometry
If same object point is identified on two images there exists epipolar plane.
The epipolar plane contais:

\begin{itemize}
\item two rays which connects object point, with image points and camera centers in both images
\item baseline, which connects projection centers of images
\item epipoles, which are defined as intersection of image planes with baseline
\item epipolar lines, which goes through epipoles and image points. It belongs into image plane, 
      becouse both points are inside this plane.  
\end{itemize}


The other relation of which comes form epipolar geomatry are:

\begin{itemize}
\item All epipolar lines in image intersects in epipole. If two images are only translated along the baseline 
     the epipole is in infinity.
\item Every image point forms epipolar line in the other image. It allows to reduce space of possible occurence of the point 
      from \eucl{2}, \eucl{1} of epipolar line, without any other information about the point. 

\end{itemize}

Algebraic derivation will be done starting with description of ray in camera coordinate system.
The ray can be defined by two points.

On of them can be projection center )$\ehvect{C}$, and 
image point $\ehvect{x}$ transform into camera coordinate system using psedo inverse matrix $\ematr{P}^{+}$, 
where $\ematr{P}\ematr{P}^{+} = \ematr{I}$. After the transformation of x, direciton vector of ray is get.
Using equation \eqref{eq:hline}, it is possible to express ray as:

\begin{equation}
\ematr{X}(\escal{\lambda}) = \ematr{P}^{+}\ehvect{x} + \escal{\lambda}\ehvect{C}
\end{equation}

In second camera with camera matrix \ematr{P'} image point coordiantes are:

\begin{equation}
\ehvect{x'} =  \ematr{P'}\ematr{P}\ehvect{x}
\end{equation}

and projective center of first camera is projected into epipole: 

\begin{equation}
\ehvect{e'} =  \ematr{P'}\ehvect{C}
\end{equation}

Having two points in image coordinate system of second camera it is possible to define epipolar line using \eqref{eq:def_hline_pts} as:

\begin{equation}
\ehvect{l'_{e}} =  \ehvect{e'} \times \ehvect{x'} = \ematr{P'}\ehvect{C} \times \ematr{P'}\ematr{P}\ehvect{x}
\end{equation}

Cross product of two vectors \evect{a} and \evect{b} can be also written as multiplication of matrix and vector:

\begin{equation}
\evect{a}  \times \evect{b}  = 
\begin{pmatrix}
   & 0      & \escal{a_{3}}   & \escal{a_{2}}\\
   & \escal{a_{3}}  & 0               & \escal{-a_{1}}\\
   & \escal{-a_{2}} & \escal{a_{1}}   & 0\\
\end{pmatrix}
\evect{b} = [a]_{\times} b
\end{equation}

Using this formulation of cross producut it is possible to write:
\begin{equation}
\ehvect{l'_{e}}  = [e']_{\times} \ematr{P'}\ematr{P}\ehvect{x}
\end{equation}

From this exuation it is evident, that matrix, which transforms image point in the first image into 
epipolar line of the second image and therfore descibes relation of two images is defined as:

\begin{equation}
\ematr{F}  = [\ehvect{e}']_{\times} \ematr{P'}\ematr{P}
\end{equation}

Matrix \ematr{F} is called fundametral matrix. 

Decomposing fundamental matrix using camera matrix expansion from \eqref{eq:p_abbr} and \eqref{eq:p_abbr}
it is possible to derive another important matrix used computer vision, which is called essential matrix \ematr{E}.

Lets suppose that firs camera matrices are defined in this way:
\begin{equation}
\ematr{P}  = \ematr{K} \ematr{[}I|0\ematr{]}
\end{equation}
It means that first camera matrix lies in the origin of object space and its orientation is same
as axis of the object system.


It's pseudo inverse matrix is:
\begin{equation}
\ematr{P}_{+} =
\begin{pmatrix}
   \ematr{K}^{-1} \\
   \evect{0^{T}} \\
\end{pmatrix}
\end{equation}

And projection center coordinates are:
\begin{equation}
\ehvect{C} =
\begin{pmatrix}
   \evect{0} \\
    1 \\
\end{pmatrix}
\end{equation}


The second camera matrix is defined in this way:
\begin{equation}
\ematr{P}'  = \ematr{K} \ematr{[}\ematr{R}|\evect{t}\ematr{]}
\end{equation}


\begin{equation}
\begin{split}
\ematr{F}  &= [\ematr{P}'\ehvect{C}]_{\times} \ematr{P}'\ematr{P} 
= [\ematr{K}' [\ematr{R}|\evect{t}]
\begin{pmatrix}
   \evect{0} \\
    1 \\
\end{pmatrix}]
_{\times} 
\ematr{K}' [\ematr{R}|\evect{t}]  
\begin{pmatrix}
   \evect{K}^{-1} \\
   \evect{0}^{T} \\
\end{pmatrix} \\
&= [\ematr{K}' \evect{t}]_{\times} \ematr{K}'\ematr{R}\ematr{K}^{-1} 
\end{split}
\end{equation}

Now comes more difficult part:
Using this formula TODO cite:
\begin{equation}
[\evect{t}]_{\times} \ematr{M} = \ematr{M}^{-T}[\ematr{M}^{-1}\evect{t}]_{\times}
\end{equation}

It is possible simply eqation:
\begin{equation}
\begin{split}
\ematr{F}  &= [\ematr{K}' \evect{t}]_{\times} \ematr{K}' \ematr{R} \ematr{K}^{-1} \\
	   &= \ematr{K}^{-T} [\ematr{K}'_{-1} \ematr{K}' \evect{t}]_{\times} \ematr{R} \ematr{K}^{-1} \\
	   &= \ematr{K}^{-T} [\evect{t}]_{\times} \ematr{R} \ematr{K}^{-1}
\end{split}
\end{equation}

This exuation reveals very important fact about fundtamental matrix thet it can be split into three  main steps.
First image point from first image is tranformed into first camera coordinate system. Result of the transformation is 
point in the infinity, which describes ray represented by image point. 
After that the ray is projected into second camera coordinate system. This is done by so called essential matrix:
\begin{equation}
	 \ematr{E}  = [\evect{t}]_{\times} \ematr{R}
\end{equation}


As a last step, the ray is projected into the second image as epipolar line.


It implies that with known exterior orientation it is possible to define essential matrix and if also interior orientation of 
cameras of both images are known it is possible to determine fundamental matrix. 


The fundametal and essential matrices are core of two view geometry in computer vision.

\subsubsection{Retrieving of exterior orientation from essential matrix}

Exterior orientation can be retrivred from essential matrix. However there exists ambiguity of signs which gives 
four possible solutions differing with sings.

The four possible exterior orientation are:

[\ematr{R}|\evect{t}],

[\ematr{R}|\evect{-t}],

[\ematr{-R}|\evect{t}],

[\ematr{-R}|\evect{-t}],

The rotation [\ematr{R}] and vector \evect{-t} can be retrived from SVD decomposition of essential matrix.

\subsubsection{Chain of essential matrices}


\subsubsection{Triangultion of points}


\section{Bundle block adjusmnet}

Bundle block adjustment is method, which uses the non-linear least square technique (\ref{sec:least}) to refine parameters 
of scene parameters. 

The main advantage of bundle block adjustment method is that the method is taking into acount whole scene and therefore using 
maximum information for adjustment of scene paramers.

Cost functions of bba least sqaures method are photogrammetric collinearity equations \label{eq:p_abbr}.

Unfurtunatelly collinearity equations are non linear, therefore non-linear least square method is used. 
As was mentioned before  (\ref{sec:non_least}) in order to method would give values of parameters, which 
satisfies minimization constrain there is needed to provide sufficiently accurate intial values.

Thanks to the least squares method BBA is very flexible in terms of choosing of parameters to be adjust. 
It is possible to select varios combinatations of interior and exterior orientation parameters and 
the collinearity model could be extended to include other parametrs e. g.  




For example three photos of a scene were taken. 3 tie points an 2 ground control points were indetified in all three photos.
Other two tie points one ground control point were identified on two photos and one grond control point and one tie point was identified 
on only one photo. 

Let say that initial values of tie points are known, but needed to be adjusted.
Coordinates of ground control points are known, therefore they do not have to be adjusted. 

For every tie point identified on image it is needed to adjust three parameters (object coordinates).
However using two collinearity equation only two measurment are fetched done, 
therefore in order to be tie point used in adjustment it has to be ideitified at least in two images. 

Because ground control point does not take any new adjusted parameters, it is possible to use gcp which are identified only in one image. 

In the example we have 3*3 adjusted parameters and 3 * 3 * 2 measurements for every tie point,  
1 * 3 * 3 measurements for every gcp indetified in all three photos. 

We have 2 * 3 adjusted parameters and 2 * 2 * 2 for every tie point,  
1 * 2 * 3 measurements for every gcp visible in two photos.

Tie point identified only in one photo can not be used because number of measuremtns is lesser (2) than number of adjusted parameters (3).
The ground control point gives two measurements.


Overall score is 15 adjusted parameters and 43 measurments paramets.

Thanks to 28 redundnt measurements it is possible to add into adjusted parameters exterior orientation of images (+ 3 * 6) and it is possible to
add some of the interior parameters into adustment.  

\section{Relative orientation}

Process of determination of two cameras exterior orientaiton in arbitrary space is called relative orientation.


\chapter{Analytical part}

In this part there are describe reasons, which led into development of process chain which was implemented in 
this thesis. All methods of the processing chain which are mentioned in this chapter, where described in 
previous theoretical chapter.


\section{General solution of UAV bundle block adjustment}


Few years ago in late 1990s it seemed that prolem of approximate values for BBA had been solved. 
In this times BBA was mainly employed in aerial photogrammetry. Mapping mission using aerial 
photogrammetry are done by experts and costs of this kind of project is very high, because
it requaires rent of airplane. The development of GPS and IMU uints, which allows to determine 
postion repectively and orientation with propelly set up and calibrated camera and GPS/IMU 
units allowed to get very precise approximate values of exterior orientation, which had been
much more difficult problem.  
  
As it was already mentioned nowadays there is strong demand for usage of these tools by non-experts 
Therefore the solution must be simple enough to be used by these kind of users. 

This is the main requiremnt on solution, which was set up at the beginning of the thesis.

To make this requiremement of simplicity more specific the user should be able to used it having 
just camera and UAV. No other hardware device should be needed. All other steps should be done 
using open source software which can everybody download.

Also the solution should be flexible enough to process aditional data, which may known, 
as GCPs approximate postion of cameras etc.With this aditional data it is possible 
to increase accuracy of results and decrease number of  BBA iterations needed for finding 
global minimum.

As consequance of these contraints, the minumum input into processing chain can be just photos
of the scene and other parameters, which can be derived from information on specific kind of photos.

The another requiremnt is two make the process as much automatic as possbile, because if it would require 
lot of time of user to go through the processing chain nobody would use it.

To fulfil these requiremnts the final major parts of the solution can be devided into this prcessing chain:

\begin{itemize}
\item Camera calibration - during camera calibration processs, the interior orientation parameters 
			   are determined.
		             Interior orientaiton can be computed 
			     from multiple photos of 2D or 3D test fields. As the test field it can be used 
			     paper with printed pattern, which can everybody do. 
				
\item Obtaining approximate exterior orienation parameters - solution should work also with photos 
which does not have known exterior orienation paametrs at all. It means 
that GPS and IMU unit does not have to be mounted on UAV and calibrated with camera.
  

\item Refine the scene parameters using BBA - As input to the BBA aproximate obtainded exterior orientation
and interior orientation from camera calibration are used.

\end{itemize}

\section{Camera calibration}




\section{Aproximate exterior orienation}

If only information which is available about the scene is interior orientaiton parameters and the photos, 
only way how to obtain aproximate exterior orientation of cameras and recover the scene structure is 
by identification of some tie points in the photos. If the scene is taken from one image, there is no way how
to retrive exterior orientation of camera if no additional information about scene e. g. gcp is not known. 
The same thing applies for photos taken with  unsufficient overlap because it is not possible to 
identify enough tie points. The last constraint is that the photos can not be taken from same position.

\subsection{Tie points identification}

The simpleist way from developer point of view is to force user to indentify tie points on 
the photographs manually. However users would have to waste a lot of time on such a 
dull task. Fortunatelly there  exist algorithms, which are able to identify tie points on images
automatically. The state of the art algorithms are SIFT \cite{wiki:SIFT} and SURF \cite{wiki:SURF}.
These algorithms allows to indentify tie points on photos, whithout need to include any other information.
Unfortunatelly both algorithms are patented in USA, therfore they use should be avoided open source community.

Recently new BRISK \cite{leutenegger2011brisk} and FREAK \cite{alahi2012freak} algorithmd 
has been developed, which does not have patent restriction. 

Another problem of tie points matching is determination of photos order, to identify photos with some overlap. 
If photos are taken in strip, it is easy to deal only with neighbour photos to identify the tie points.

If there is no apriori information about image order, it is as in all cases possible to use brute force and 
try to find match image with all other images.



This part was covered in this thesis just with this short overview, becouse this topic is so big 
that it would be possible to write thesis focused only on feature extraction or building of visibility map.



\chapter{Practical part}

\chapter{Future development}

\chapter{Conclusion}


\section{References}
\nobibliography{BP}
\bibliographystyle{plain}
\bibentry{Hartley2004}

\bibentry{wiki:SIFT}

\bibentry{wiki:SURF}

\bibentry{leutenegger2011brisk}
\end{document}}
